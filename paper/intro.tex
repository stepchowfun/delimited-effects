\section{Introduction}

In the two decades since they were introduced by \citet{wadler89}, \emph{type classes} have proven a versatile apparatus for structuring programs. Consider a class intended to characterize types \(\eA\) for which a monoid can be defined:
\begin{flalign*}
  & \eClass{\tMonoid{\eA}} & \\
  & \quad \tAnno{\eIdentity}{\eA} & \\
  & \quad \tAnno{\eCombine}{\tArrow{\eA}{\tArrow{\eA}{\eA}}} &
\end{flalign*}
Programmers can use this class to define polymorphic functions over all monoids:
\begin{flalign*}
  & \tAnno{\eSquare}{\tMonoid{\eA} \Rightarrow \tArrow{\eA}{\eA}} & \\
  & \eApp{\eSquare}{\eX} = \eApp{\eApp{\eCombine}{\eX}}{\eX} &
\end{flalign*}
Finally, an \emph{instance} of the class can be provided for a specific type:
\begin{flalign*}
  & \eInstance{\tMonoid{\integer}} & \\
  & \quad \eIdentity = \numberName{1} & \\
  & \quad \eCombine = \eMultiply &
\end{flalign*}
where \(\tAnno{\numberName{1}}{\integer}\) and \(\tAnno{\eMultiply}{\tArrow{\integer}{\tArrow{\integer}{\integer}}}\). The instance is a witness to justify the well-definedness of expressions like \(\eApp{\eSquare}{\numberName{3}}\).

More recently, algebraic effects \citep{plotkin03} and handlers \citep{plotkin09} have become an attractive alternative to monads for modeling computational effects due to their compositionality.
