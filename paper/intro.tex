\section{Introduction}

We call a monad \(\left(T, \eta, \mu\right)\) in a category \(\mathbb{C}\) \emph{algebraic} if every natural transformation \(\varepsilon : T \rightarrow 1_\mathbb{C}\) makes the diagrams

\medskip
\noindent\begin{minipage}{\dimexpr.5\textwidth-.5\columnsep}
  \centering
  \begin{tikzcd}[row sep=huge, column sep=huge]
    1_\mathbb{C} \arrow[r, "\eta"] \arrow[rd, "1"]
      & T \arrow[d, "\varepsilon"] \\
    & 1_\mathbb{C}
  \end{tikzcd}
\end{minipage}\begin{minipage}{\dimexpr.5\textwidth-.5\columnsep}
  \centering
  \begin{tikzcd}[row sep=huge, column sep=huge]
    T^2
        \arrow[r, "\mu"]
        \arrow[d, "\varepsilon T"]
      & T
        \arrow[d, "\varepsilon"]
        \arrow[dl, equal] \\
    T
        \arrow[r, "\varepsilon"]
      & 1_\mathbb{C}
  \end{tikzcd}
\end{minipage}
\medskip

\noindent commute. If \(\left(T, \eta, \mu\right)\) and \(\left(U, \delta, \nu\right)\) are algebraic monads, then any pair of natural transformations \(\varepsilon : T \rightarrow 1_\mathbb{C}\) and \(\theta : U \rightarrow 1_\mathbb{C}\) induces a composite algebraic monad \(\left(T U, \eta U \circ \delta, \theta TU \circ \varepsilon UTU\right)\). Furthermore, \(\delta T \circ \eta \circ \theta \circ \varepsilon U : TU \rightarrow UT\) is a natural isomorphism, which is to say composition of algebraic monads is commutative.
