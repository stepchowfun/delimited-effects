\section{The calculus}

  We adopt the familiar conventions that alpha-equivalent expressions are interchangeable and that bound variables are distinct from free variables in all contexts.

  \subsection{Syntax}

    \begin{figure}[H]
      \begin{center}
        \begin{tabular}{r l c l}
          term variables & \(\eVar\) & \(\Coloneqq\) & \(
            \eX                                             \enspace | \enspace
            \eY                                             \enspace | \enspace
            \ldots                                          \) \\
          type variables & \(\tVar\) & \(\Coloneqq\) & \(
            \alpha                                          \enspace | \enspace
            \beta                                           \enspace | \enspace
            \ldots                                          \) \\
          terms & \(\term\) & \(\Coloneqq\) & \(
            \eVar                                           \enspace | \enspace
            \eAbs{\eVar}{\type}{\term}                      \enspace | \enspace
            \eApp{\term}{\term}                             \enspace | \enspace
            \eTAbs{\tVar}{\kind}{\term}                     \enspace | \enspace
            \eTApp{\term}{\type}                            \) \\
          & & \(|\) & \(
            \eReify{\tVar}{\term}{\term}                    \enspace | \enspace
            \eReflect{\tVar}                                \) \\
          & & \(|\) & \(
            \ePure{\term}                                   \enspace | \enspace
            \eRun{\term}                                    \enspace | \enspace
            \eDo{\eVar}{\term}{\term}                       \) \\ \\
          types & \(\type\) & \(\Coloneqq\) & \(
            \tVar                                           \enspace | \enspace
            \tArrow{\type}{\type}                           \enspace | \enspace
            \tForAll{\tVar}{\kind}{\type}                   \enspace | \enspace
            \tComputation{\type}{\type}                     \enspace | \enspace
            \tPure                                          \) \\ \\
          kinds & \(\kind\) & \(\Coloneqq\) & \(
            \kType                                          \enspace | \enspace
            \kEffect                                        \enspace | \enspace
            \kWitness{\type}                                \) \\ \\
          contexts & \(\context\) & \(\Coloneqq\) & \(
            \cEmpty                                         \enspace | \enspace
            \cEExtend{\context}{\eVar}{\type}               \enspace | \enspace
            \cTExtend{\context}{\tVar}{\kind}               \)
        \end{tabular}
      \end{center}

      \caption{Syntax}
      \label{fig:syntax}
    \end{figure}

  \subsection{Typing}

    \begin{figure}[H]
      \begin{center}
        \framebox{\(\hasType{\context}{\term}{\type}\)}
      \end{center}

      \medskip

      \begin{prooftree}
          \AxiomC{\(\apply{\context}{\eVar} = \type\)}
        \RightLabel{(\textsc{T-Var})}
        \UnaryInfC{\(\hasType{\context}{\eVar}{\type}\)}
      \end{prooftree}

      \begin{prooftree}
          \AxiomC{\(\hasType{\cEExtend{\context}{\eVar}{\type_1}}{\term}{\type_2}\)}
          \AxiomC{\(\hasKind{\context}{\type_1}{\kType}\)}
        \RightLabel{(\textsc{T-Abs})}
        \BinaryInfC{\(\hasType{\context}{\parens{\eAbs{\eVar}{\type_1}{\term}}}{\tArrow{\type_1}{\type_2}}\)}
      \end{prooftree}

      \begin{prooftree}
          \AxiomC{\(\hasType{\context}{\term_1}{\tArrow{\type_1}{\type_2}}\)}
          \AxiomC{\(\hasType{\context}{\term_2}{\type_1}\)}
        \RightLabel{(\textsc{T-App})}
        \BinaryInfC{\(\hasType{\context}{\eApp{\term_1}{\term_2}}{\type_2}\)}
      \end{prooftree}

      \begin{prooftree}
          \AxiomC{\(\hasType{\cTExtend{\context}{\tVar}{\kind}}{\term}{\type}\)}
          \AxiomC{\(\tVar \notin \dom{\context}\)}
        \RightLabel{(\textsc{T-TAbs})}
        \BinaryInfC{\(\hasType{\context}{\parens{\eTAbs{\tVar}{\kind}{\term}}}{\parens{\tForAll{\tVar}{\kind}{\type}}}\)}
      \end{prooftree}

      \begin{prooftree}
          \AxiomC{\(\hasType{\context}{\term}{\parens{\tForAll{\tVar}{\kind}{\type_1}}}\)}
          \AxiomC{\(\hasKind{\context}{\type_2}{\kind}\)}
        \RightLabel{(\textsc{T-TApp})}
        \BinaryInfC{\(\hasType{\context}{\eTApp{\term}{\type_2}}{\substitute{\type_1}{\tVar}{\type_2}}\)}
      \end{prooftree}

      \begin{prooftree}
          \AxiomC{\(\hasType{\context}{\term_1}{\type_1}\)}
          \AxiomC{\(\hasType{\cTExtend{\context}{\tVar}{\kWitness{\type_1}}}{\term_2}{\type_2}\)}
          \AxiomC{\(\hasKind{\context}{\type_2}{\kType}\)}
          \AxiomC{\(\tVar \notin \dom{\context}\)}
        \RightLabel{(\textsc{T-Reify})}
        \QuaternaryInfC{\(\hasType{\context}{\eReify{\tVar}{\term_1}{\term_2}}{\type_2}\)}
      \end{prooftree}

      \begin{prooftree}
          \AxiomC{\(\hasKind{\context}{\tVar}{\kWitness{\type}}\)}
        \RightLabel{(\textsc{T-Reflect})}
        \UnaryInfC{\(\hasType{\context}{\eReflect{\tVar}}{\type}\)}
      \end{prooftree}

      \begin{prooftree}
          \AxiomC{\(\hasType{\context}{\term_1}{\tComputation{\type_1}{\type}}\)}
          \AxiomC{\(\hasType{\cEExtend{\context}{\eVar}{\type_1}}{\term_2}{\tComputation{\type_2}{\type}}\)}
        \RightLabel{(\textsc{T-Do})}
        \BinaryInfC{\(\hasType{\context}{\eDo{\eVar}{\term_1}{\term_2}}{\tComputation{\type_2}{\type}}\)}
      \end{prooftree}

      \begin{prooftree}
          \AxiomC{\(\hasType{\context}{\term}{\type}\)}
          \AxiomC{\(\hasKind{\context}{\type}{\kEffect}\)}
        \RightLabel{(\textsc{T-Pure})}
        \BinaryInfC{\(\hasType{\context}{\ePure{\term}}{\tComputation{\type}{\type}}\)}
      \end{prooftree}

      \begin{prooftree}
          \AxiomC{\(\hasType{\context}{\term}{\tComputation{\type}{\tPure}}\)}
        \RightLabel{(\textsc{T-Run})}
        \UnaryInfC{\(\hasType{\context}{\eRun{\term}}{\type}\)}
      \end{prooftree}

      \caption{Typing rules}
      \label{fig:typing}
    \end{figure}

    \begin{figure}[H]
      \begin{center}
        \framebox{\(\hasKind{\context}{\type}{\kind}\)}
      \end{center}

      \medskip

      \begin{prooftree}
          \AxiomC{\(\apply{\context}{\tVar} = \kind\)}
        \RightLabel{(\textsc{K-Var})}
        \UnaryInfC{\(\hasKind{\context}{\tVar}{\kind}\)}
      \end{prooftree}

      \begin{prooftree}
          \AxiomC{\(\hasKind{\context}{\type_1}{\kType}\)}
          \AxiomC{\(\hasKind{\context}{\type_2}{\kType}\)}
        \RightLabel{(\textsc{K-Arrow})}
        \BinaryInfC{\(\hasKind{\context}{\tArrow{\type_1}{\type_2}}{\kType}\)}
      \end{prooftree}

      \begin{prooftree}
          \AxiomC{\(\hasKind{\cTExtend{\context}{\tVar}{\kind}}{\type}{\kType}\)}
        \RightLabel{(\textsc{K-ForAll})}
        \UnaryInfC{\(\hasKind{\context}{\parens{\tForAll{\tVar}{\kind}{\type}}}{\kType}\)}
      \end{prooftree}

      \begin{prooftree}
          \AxiomC{\(\hasKind{\context}{\type}{\kType}\)}
          \AxiomC{\(\hasKind{\context}{\type}{\kEffect}\)}
        \RightLabel{(\textsc{K-Computation})}
        \BinaryInfC{\(\hasKind{\context}{\tComputation{\type}{\type}}{\kType}\)}
      \end{prooftree}

      \begin{prooftree}
          \AxiomC{}
        \RightLabel{(\textsc{K-Pure})}
        \UnaryInfC{\(\hasKind{\context}{\tPure}{\kEffect}\)}
      \end{prooftree}

      \caption{Kinding rules}
      \label{fig:kinding}
    \end{figure}

  \subsection{Semantics}

    Consider the program
    \[
      \eReify{\alpha}{\term}{\eAbs{\eX}{\type_1}{\eTApp{\parens{\eTAbs{\beta}{\kWitness{\type_2}}{\eReflect{\beta}}}}{\alpha}}},
    \]
    which is well-typed when \(\hasType{\cEmpty}{\term}{\type_2}\). Intuition suggests the (weak) normal form \(\eAbs{\eX}{\type_1}{\term}\). What might be the first step this program takes toward that normal form? One is tempted to eliminate the type application:
    \[
      \eReify{\alpha}{\term}{\eAbs{\eX}{\type_1}{\eReflect{\alpha}}}.
    \]
    But that would be reduction under a lambda abstraction, which is ordinarily forbidden in lambda calculi intended for programming. Indeed, this modest calculus defies a straightforward operational semantics.

    However, the system admits a type-directed translation into call-by-value System F, shown in Figure~\ref{fig:semantics}.

    \begin{figure}[H]
      \ifnoacm
        \small
      \fi

      \begin{center}
        \framebox{\(\translatesTo{\context}{\term}{\term}\)}
        \framebox{\(\translatesTo{\context}{\type}{\type}\)}
        \framebox{\(\translatesTo{\context}{\type}{\term}\)}
      \end{center}

      \medskip

      \begin{center}
          \AxiomC{}
        \RightLabel{(\textsc{S-Var})}
        \UnaryInfC{\(\translatesTo{\context}{\eVar}{\eVar}\)}
        \DisplayProof
        \qquad
          \AxiomC{\(\translatesTo{\context}{\term_1}{\term_3}\)}
          \AxiomC{\(\translatesTo{\context}{\term_2}{\term_4}\)}
        \RightLabel{(\textsc{S-App})}
        \BinaryInfC{\(\translatesTo{\context}{\eApp{\term_1}{\term_2}}{\eApp{\term_3}{\term_4}}\)}
        \DisplayProof
      \end{center}

      \begin{center}
          \AxiomC{\(\translatesTo{\context}{\type_1}{\type_2}\)}
          \AxiomC{\(\translatesTo{\context}{\term_1}{\term_2}\)}
        \RightLabel{(\textsc{S-Abs1})}
        \BinaryInfC{\(\translatesTo{\context}{\eAbs{\eVar}{\type_1}{\term_1}}{\eAbs{\eVar}{\type_2}{\term_2}}\)}
        \DisplayProof
        \qquad
          \AxiomC{\(\translatesTo{\context}{\type}{\term_3}\)}
          \AxiomC{\(\translatesTo{\context}{\term_1}{\term_2}\)}
        \RightLabel{(\textsc{S-Abs2})}
        \BinaryInfC{\(\translatesTo{\context}{\eAbs{\eVar}{\type}{\term_1}}{\eAbs{\eVar}{\tVoid}{\term_2}}\)}
        \DisplayProof
      \end{center}

      \begin{center}
          \AxiomC{\(\translatesTo{\context}{\term_1}{\term_2}\)}
        \RightLabel{(\textsc{S-TAbs1})}
        \UnaryInfC{\(\translatesTo{\context}{\eTAbs{\tVar}{\kType}{\term_1}}{\eTAbsNoKind{\tVar}{\term_2}}\)}
        \DisplayProof
        \qquad
          \AxiomC{\(\translatesTo{\context}{\term_1}{\term_2}\)}
        \RightLabel{(\textsc{S-TAbs2})}
        \UnaryInfC{\(\translatesTo{\context}{\eTAbs{\tVar}{\kEffect}{\term_1}}{\eTAbsNoKind{\tVar}{\term_2}}\)}
        \DisplayProof
      \end{center}

      \begin{center}
          \AxiomC{\(\translatesTo{\context}{\type_1}{\type_2}\)}
          \AxiomC{\(\translatesTo{\context}{\term_1}{\term_2}\)}
        \RightLabel{(\textsc{S-TAbs3})}
        \BinaryInfC{\(\translatesTo{\context}{\eTAbs{\tVar}{\kWitness{\type_1}}{\term_1}}{\eAbs{\eVar_{\tVar}}{\type_2}{\term_2}}\)}
        \DisplayProof
        \qquad
          \AxiomC{\(\translatesTo{\context}{\type}{\term_3}\)}
          \AxiomC{\(\translatesTo{\context}{\term_1}{\term_2}\)}
        \RightLabel{(\textsc{S-TAbs4})}
        \BinaryInfC{\(\translatesTo{\context}{\eTAbs{\tVar}{\kWitness{\type}}{\term_1}}{\eAbs{\eVar_{\tVar}}{\tVoid}{\term_2}}\)}
        \DisplayProof
      \end{center}

      \begin{center}
          \AxiomC{\(\translatesTo{\context}{\term_1}{\term_2}\)}
          \AxiomC{\(\translatesTo{\context}{\type_1}{\type_2}\)}
        \RightLabel{(\textsc{S-TApp1})}
        \BinaryInfC{\(\translatesTo{\context}{\eTApp{\term_1}{\type_1}}{\eTApp{\term_2}{\type_2}}\)}
        \DisplayProof
        \qquad
          \AxiomC{\(\translatesTo{\context}{\term_1}{\term_2}\)}
          \AxiomC{\(\translatesTo{\context}{\type}{\term_3}\)}
        \RightLabel{(\textsc{S-TApp2})}
        \BinaryInfC{\(\translatesTo{\context}{\eTApp{\term_1}{\type}}{\eApp{\term_2}{\term_3}}\)}
        \DisplayProof
      \end{center}

      \begin{center}
          \AxiomC{\(\translatesTo{\context}{\term_1}{\term_3}\)}
          \AxiomC{\(\translatesTo{\context}{\term_2}{\term_4}\)}
        \RightLabel{(\textsc{S-Reify})}
        \BinaryInfC{\(\translatesTo{\context}{\eReify{\tVar}{\term_1}{\term_2}}{\substitute{\term_4}{\eVar_{\tVar}}{\term_3}}\)}
        \DisplayProof
        \qquad
          \AxiomC{}
        \RightLabel{(\textsc{S-Reflect})}
        \UnaryInfC{\(\translatesTo{\context}{\eReflect{\tVar}}{\eVar_{\tVar}}\)}
        \DisplayProof
      \end{center}

      \begin{center}
          \AxiomC{\(\translatesTo{\context}{\term_1}{\term_2}\)}
        \RightLabel{(\textsc{S-Pure})}
        \UnaryInfC{\(\translatesTo{\context}{\ePure{\term_1}}{\eAbs{\wildcard}{\tUnit}{\term_2}}\)}
        \DisplayProof
        \qquad
          \AxiomC{\(\translatesTo{\context}{\term_1}{\term_2}\)}
        \RightLabel{(\textsc{S-Run})}
        \UnaryInfC{\(\translatesTo{\context}{\eRun{\term_1}}{\eApp{\term_2}{\eUnit}}\)}
        \DisplayProof
      \end{center}

      \begin{prooftree}
          \AxiomC{\(\hasType{\context}{\term_1}{\type_1}\)}
          \AxiomC{\(\translatesTo{\context}{\type_1}{\type_2}\)}
          \AxiomC{\(\translatesTo{\context}{\term_1}{\term_3}\)}
          \AxiomC{\(\translatesTo{\context}{\term_2}{\term_4}\)}
        \RightLabel{(\textsc{S-Do})}
        \QuaternaryInfC{\(\translatesTo{\context}{\eDo{\eVar}{\term_1}{\term_2}}{\eApp{\parens{\eAbs{\eVar}{\type_2}{\term_4}}}{\term_3}}\)}
      \end{prooftree}

      \begin{center}
          \AxiomC{\(\hasKind{\context}{\tVar}{\kType}\)}
        \RightLabel{(\textsc{S-TVar1})}
        \UnaryInfC{\(\translatesTo{\context}{\tVar}{\tVar}\)}
        \DisplayProof
        \qquad
          \AxiomC{\(\hasKind{\context}{\tVar}{\kWitness{\type}}\)}
        \RightLabel{(\textsc{S-TVar2})}
        \UnaryInfC{\(\translatesTo{\context}{\tVar}{\eVar_{\tVar}}\)}
        \DisplayProof
      \end{center}

      \begin{center}
          \AxiomC{\(\translatesTo{\context}{\type_1}{\type_3}\)}
          \AxiomC{\(\translatesTo{\context}{\type_2}{\type_4}\)}
        \RightLabel{(\textsc{S-Arrow1})}
        \BinaryInfC{\(\translatesTo{\context}{\tArrow{\type_1}{\type_2}}{\tArrow{\type_3}{\type_4}}\)}
        \DisplayProof
        \qquad
          \AxiomC{\(\translatesTo{\context}{\type_1}{\type_3}\)}
          \AxiomC{\(\translatesTo{\context}{\type_2}{\term}\)}
        \RightLabel{(\textsc{S-Arrow2})}
        \BinaryInfC{\(\translatesTo{\context}{\tArrow{\type_1}{\type_2}}{\tArrow{\type_3}{\tVoid}}\)}
        \DisplayProof
      \end{center}

      \begin{center}
          \AxiomC{\(\translatesTo{\context}{\type_1}{\term}\)}
          \AxiomC{\(\translatesTo{\context}{\type_2}{\type_3}\)}
        \RightLabel{(\textsc{S-Arrow3})}
        \BinaryInfC{\(\translatesTo{\context}{\tArrow{\type_1}{\type_2}}{\tArrow{\tVoid}{\type_3}}\)}
        \DisplayProof
        \qquad
          \AxiomC{\(\translatesTo{\context}{\type_1}{\term_1}\)}
          \AxiomC{\(\translatesTo{\context}{\type_2}{\term_2}\)}
        \RightLabel{(\textsc{S-Arrow4})}
        \BinaryInfC{\(\translatesTo{\context}{\tArrow{\type_1}{\type_2}}{\tArrow{\tVoid}{\tVoid}}\)}
        \DisplayProof
      \end{center}

      \begin{center}
          \AxiomC{\(\translatesTo{\context}{\type_1}{\type_2}\)}
        \RightLabel{(\textsc{S-ForAll1})}
        \UnaryInfC{\(\translatesTo{\context}{\tForAll{\tVar}{\kType}{\type_1}}{\tForAllNoKind{\tVar}{\type_2}}\)}
        \DisplayProof
        \qquad
          \AxiomC{\(\translatesTo{\context}{\type}{\term}\)}
        \RightLabel{(\textsc{S-ForAll2})}
        \UnaryInfC{\(\translatesTo{\context}{\tForAll{\tVar}{\kType}{\type}}{\tForAllNoKind{\tVar}{\tVoid}}\)}
        \DisplayProof
      \end{center}

      \begin{center}
          \AxiomC{\(\translatesTo{\context}{\type_1}{\type_3}\)}
          \AxiomC{\(\translatesTo{\context}{\type_2}{\type_4}\)}
        \RightLabel{(\textsc{S-ForAll3})}
        \BinaryInfC{\(\translatesTo{\context}{\tForAll{\tVar}{\kWitness{\type_1}}{\type_2}}{\tArrow{\type_3}{\type_4}}\)}
        \DisplayProof
        \qquad
          \AxiomC{\(\translatesTo{\context}{\type_1}{\type_3}\)}
          \AxiomC{\(\translatesTo{\context}{\type_2}{\term}\)}
        \RightLabel{(\textsc{S-ForAll4})}
        \BinaryInfC{\(\translatesTo{\context}{\tForAll{\tVar}{\kWitness{\type_1}}{\type_2}}{\tArrow{\type_3}{\tVoid}}\)}
        \DisplayProof
      \end{center}

      \begin{center}
          \AxiomC{\(\translatesTo{\context}{\type_1}{\term}\)}
          \AxiomC{\(\translatesTo{\context}{\type_2}{\type_3}\)}
        \RightLabel{(\textsc{S-ForAll5})}
        \BinaryInfC{\(\translatesTo{\context}{\tForAll{\tVar}{\kWitness{\type_1}}{\type_2}}{\tArrow{\tVoid}{\type_3}}\)}
        \DisplayProof
        \qquad
          \AxiomC{\(\translatesTo{\context}{\type_1}{\term_1}\)}
          \AxiomC{\(\translatesTo{\context}{\type_2}{\term_2}\)}
        \RightLabel{(\textsc{S-ForAll6})}
        \BinaryInfC{\(\translatesTo{\context}{\tForAll{\tVar}{\kWitness{\type_1}}{\type_2}}{\tArrow{\tVoid}{\tVoid}}\)}
        \DisplayProof
      \end{center}

      \begin{center}
          \AxiomC{\(\translatesTo{\context}{\type_1}{\type_2}\)}
        \RightLabel{(\textsc{S-Computation})}
        \UnaryInfC{\(\translatesTo{\context}{\tComputation{\type_1}{\type}}{\tArrow{\tUnit}{\type_2}}\)}
        \DisplayProof
          \AxiomC{\(\hasKind{\context}{\type}{\kEffect}\)}
        \RightLabel{(\textsc{S-Effect})}
        \UnaryInfC{\(\translatesTo{\context}{\type}{\tUnit}\)}
        \DisplayProof
      \end{center}

      \caption{Elaboration into System F}
      \label{fig:semantics}
    \end{figure}
