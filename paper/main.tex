%%%%%%%%%%%%%%%%%%%%%%%%%%%%%%%%%%
% Document settings and packages %
%%%%%%%%%%%%%%%%%%%%%%%%%%%%%%%%%%

\newif\ifuseacm
\newcommand\ifnoacm{\ifuseacm\else}
\newcommand\ifacm{\ifuseacm}

% Comment out this line to disable the acmart document class
% \useacmtrue

\ifnoacm
  \documentclass[12pt]{article}
\fi

\ifacm
  \documentclass[format=acmsmall, review=false, screen=true]{acmart}
\fi

\usepackage[T1]{fontenc} % The font encoding
\usepackage[varqu]{zi4} % Inconsolata font with straight quotes
\usepackage{amssymb} % For \varnothing
\usepackage{bussproofs} % For inference rules
\usepackage{float} % For the [H] option in \begin{figure}[H]. NOTE: Stop using this when we have more content
\usepackage{mathtools} % For \Coloneqq
\usepackage{stackengine} % For stacking axioms
\usepackage{stmaryrd} % For \llbracket and \rrbracket

\ifnoacm
  \usepackage[margin=1in]{geometry} % For setting the margin
  \usepackage{natbib} % For citations and references
\fi

%%%%%%%%%%%%
% Metadata %
%%%%%%%%%%%%

\title{Type and effect classes}

\ifnoacm
  \date{}
\fi

\author{Stephan Boyer}

\ifacm
  \orcid{0000-0003-4599-2683}
  \affiliation{%
    \institution{Airbnb, Inc.}
    \streetaddress{888 Brannan St.}
    \city{San Francisco}
    \state{CA}
    \postcode{94103}
    \country{USA}}
  \email{stephan@stephanboyer.com}

  \acmJournal{PACMPL}
  \acmVolume{3}
  \acmNumber{POPL}
  \acmArticle{39}
  \acmYear{2019}
  \acmMonth{3}
  \acmDOI{0000001.0000001}

  \setcopyright{acmlicensed}
  \copyrightyear{2018}

  \received{February 2018}
  \received[revised]{March 2018}
  \received[accepted]{June 2018}

% For some reason, LaTeX throws an error if this is indented.
% This was generated by: http://dl.acm.org/ccs.cfm
\begin{CCSXML}
  <ccs2012>
    <concept>
      <concept_id>10003752.10010124.10010125.10010130</concept_id>
      <concept_desc>Theory of computation~Type structures</concept_desc>
      <concept_significance>500</concept_significance>
    </concept>
    <concept>
      <concept_id>10003752.10010124.10010131.10010134</concept_id>
      <concept_desc>Theory of computation~Operational semantics</concept_desc>
      <concept_significance>300</concept_significance>
    </concept>
    <concept>
      <concept_id>10011007.10011006.10011008.10011009.10011012</concept_id>
      <concept_desc>Software and its engineering~Functional languages</concept_desc>
      <concept_significance>500</concept_significance>
    </concept>
  </ccs2012>
\end{CCSXML}

  \ccsdesc[500]{Theory of computation~Type structures}
  \ccsdesc[300]{Theory of computation~Operational semantics}
  \ccsdesc[500]{Software and its engineering~Functional languages}

  \keywords{Type classes, effect classes, algebraic effects, effect handlers, reflection}
\fi

%%%%%%%%%%%%%%%
% The writeup %
%%%%%%%%%%%%%%%

% Typesetting
\newcommand\constructorName[1]{\textsf{\textsc{#1}}}
\newcommand\keyword[1]{\textsf{\textbf{#1}}}
\newcommand\numberName[1]{\textsf{#1}}
\newcommand\variableName[1]{\textsf{#1}}

% Misc
\newcommand\apply[2]{#1\parens{#2}}
\newcommand\dom[1]{\apply{\operatorname{dom}}{#1}}
\newcommand\kAnno[2]{#1 : #2}
\newcommand\parens[1]{\mathopen{}\left( #1 \right)\mathclose{}}
\newcommand\substitute[3]{#1 \left[ #3 / #2 \right]}
\newcommand\tAnno[2]{#1 : #2}

% Terms
\newcommand\term{t}
\newcommand\eVar{x}
\newcommand\eAbs[3]{\lambda \tAnno{#1}{#2} \; . \; #3}
\newcommand\eApp[2]{#1 \; #2}
\newcommand\eTAbs[3]{\Lambda \kAnno{#1}{#2} \; . \; #3}
\newcommand\eTAbsNoKind[2]{\Lambda #1 \; . \; #2}
\newcommand\eTApp[2]{#1 \; #2}
\newcommand\eReify[3]{\keyword{reify} \; #1 \leftarrow #2 \; \keyword{in} \; #3}
\newcommand\eReflect[1]{\keyword{reflect} \; #1}
\newcommand\ePure[1]{\keyword{pure} \; #1}
\newcommand\eRun[1]{\keyword{run} \; #1}
\newcommand\eDo[3]{\keyword{do} \; #1 \leftarrow #2 \; \keyword{in} \; #3}
\newcommand\eUnit{\parens{}}

% Types and rows
\newcommand\type{\tau}
\newcommand\tVar{a}
\newcommand\tArrow[2]{#1 \rightarrow #2}
\newcommand\tForAll[3]{\forall \kAnno{#1}{#2} \; . \; #3}
\newcommand\tForAllNoKind[2]{\forall #1 \; . \; #2}
\newcommand\tComputation[2]{#1 \; ! \; #2}
\newcommand\tPure{\circ}
\newcommand\tVoid{\constructorName{0}}
\newcommand\tUnit{\constructorName{1}}

% Kinds
\newcommand\kind{\kappa}
\newcommand\kType{\star}
\newcommand\kEffect{\diamond}
\newcommand\kWitness[1]{\left\langle #1 \right\rangle}

% Contexts
\newcommand\context{\Gamma}
\newcommand\cEmpty{\varnothing}
\newcommand\cEExtend[3]{#1, \tAnno{#2}{#3}}
\newcommand\cTExtend[3]{#1, \kAnno{#2}{#3}}

% Judgments
\newcommand\hasType[3]{#1 \vdash \tAnno{#2}{#3}}
\newcommand\hasKind[3]{#1 \vdash \kAnno{#2}{#3}}
\newcommand\translatesTo[3]{#1 \vdash #2 \rightsquigarrow #3}

% Pseudo-constructs used for exposition
\newcommand\eClass[1]{\keyword{class} \; #1 \; \keyword{where}}
\newcommand\eCombine{\variableName{combine}}
\newcommand\eData[2]{\keyword{data} \; #1 = #2}
\newcommand\eIdentity{\variableName{identity}}
\newcommand\eInstance[1]{\keyword{instance} \; #1 \; \keyword{where}}
\newcommand\eLet[3]{\keyword{let} \; #1 = #2 \; \keyword{in} \; #3}
\newcommand\eMonoid[2]{\eApp{\eApp{\monoid}{#1}}{#2}}
\newcommand\eMultiply{\variableName{multiply}}
\newcommand\eSquare{\variableName{square}}
\newcommand\eX{\variableName{x}}
\newcommand\eY{\variableName{y}}
\newcommand\eZ{\variableName{z}}
\newcommand\integer{\constructorName{Integer}}
\newcommand\monoid{\constructorName{Monoid}}
\newcommand\tApp[2]{#1 \; #2}
\newcommand\tMonoid[1]{\tApp{\monoid}{#1}}
\newcommand\wildcard{\_}


\begin{document}
  \ifacm
    \citestyle{acmauthoryear}

    % The `acmart` layout requires the abstract to be before \maketitle.
    % But the `article` layout requires the abstract to be after \maketitle.
    \begin{abstract}
      Motivated by an apparent similarity between the rules by which type class constraints and algebraic effects propagate through programs in their respective formalisms, we introduce a novel language feature capable of modeling both. Our system offers a uniform treatment of algebraic effects and type classes. Building on the idea \cite{kiselyov04} by Kiselyov and Shan of reifying terms as types and reflecting them back into terms, our system preserves coherence for local type class instances.

    \end{abstract}
  \fi

  \maketitle

  \ifnoacm
    \citestyle{authoryear}

    % The `acmart` layout requires the abstract to be before \maketitle.
    % But the `article` layout requires the abstract to be after \maketitle.
    \begin{abstract}
      Motivated by an apparent similarity between the rules by which type class constraints and algebraic effects propagate through programs in their respective formalisms, we introduce a novel language feature capable of modeling both. Our system offers a uniform treatment of algebraic effects and type classes. Building on the idea \cite{kiselyov04} by Kiselyov and Shan of reifying terms as types and reflecting them back into terms, our system preserves coherence for local type class instances.

    \end{abstract}
  \fi

  \section{Introduction}

In the two decades since they were introduced by \citet{wadler89}, \emph{type classes} have proven a versatile apparatus for structuring programs. Consider a class intended to characterize types \(\eA\) for which a monoid can be defined:
\begin{flalign*}
  & \eClass{\tMonoid{\eA}} & \\
  & \quad \tAnno{\eIdentity}{\eA} & \\
  & \quad \tAnno{\eCombine}{\tArrow{\eA}{\tArrow{\eA}{\eA}}} &
\end{flalign*}
Programmers can use this class to define polymorphic functions over all monoids:
\begin{flalign*}
  & \tAnno{\eSquare}{\tMonoid{\eA} \Rightarrow \tArrow{\eA}{\eA}} & \\
  & \eApp{\eSquare}{\eX} = \eApp{\eApp{\eCombine}{\eX}}{\eX} &
\end{flalign*}
Finally, an \emph{instance} of the class can be provided for a specific type:
\begin{flalign*}
  & \eInstance{\tMonoid{\integer}} & \\
  & \quad \eIdentity = \numberName{1} & \\
  & \quad \eCombine = \eMultiply &
\end{flalign*}
where \(\tAnno{\numberName{1}}{\integer}\) and \(\tAnno{\eMultiply}{\tArrow{\integer}{\tArrow{\integer}{\integer}}}\). The instance is a witness to justify the well-definedness of expressions like \(\eApp{\eSquare}{\numberName{3}}\).

More recently, algebraic effects \citep{plotkin03} and handlers \citep{plotkin09} have become an attractive alternative to monads for modeling computational effects due to their compositionality.

  \section{Overview}

  \subsection{Reflection and reification}

    Here's a simple example of reification and reflection.
    \[ \eReify{\tVar}{\numberName{4}}{\parens{\numberName{3} + \eReflect{\tVar}}} \]

  \subsection{Type classes with local instances}

    We can represent a type class as an ordinary data type.
    \[ \eData{\tMonoid{\tVar}}{\eMonoid{\tVar}{\parens{\tArrow{\tVar}{\tArrow{\tVar}{\tVar}}}}} \]
    For convenience, we define \emph{methods} for the class. These are functions that reflect a reified instance and project the appropriate components.
    \[ \eIdentity = \eTAbs{\tVar_1}{\kType}{\eTAbs{\tVar_2}{\kWitness{\tMonoid{\tVar_1}}}{\eLet{\eMonoid{\eVar}{\wildcard}}{\eReflect{\tVar_2}}{\eVar}}} \]
    \[ \eCombine = \eTAbs{\tVar_1}{\kType}{\eTAbs{\tVar_2}{\kWitness{\tMonoid{\tVar_1}}}{\eLet{\eMonoid{\wildcard}{\eVar}}{\eReflect{\tVar_2}}{\eVar}}} \]
    Here's a polymorphic function which ``squares'' its argument using the monoid operation:
    \[ \eSquare = \eTAbs{\tVar_1}{\kType}{\eTAbs{\tVar_2}{\kWitness{\tMonoid{\tVar_1}}}{\eAbs{\eVar}{\tVar_1}{\eApp{\eApp{\eTApp{\eTApp{\eCombine}{\tVar_1}}{\tVar_2}}{\eVar}}{\eVar}}}} \]
    Here, we create and use a local instance of the class which reifies the monoidal structure of integers under multiplication:
    \[ \eReify{\tVar}{\eMonoid{\numberName{1}}{\eMultiply}}{\eApp{\eSquare}{\numberName{3}}} \]

  \subsection{Effects and handlers}

  \section{The calculus}

  We adopt the familiar conventions that alpha-equivalent expressions are interchangeable and that bound variables are distinct from free variables in all contexts.

  \subsection{Syntax}

    \begin{figure}[H]
      \begin{center}
        \begin{tabular}{r l c l}
          terms & \(\term\) & \(\Coloneqq\) & \(
            \eVar                                           \enspace | \enspace
            \eAbs{\eVar}{\type}{\term}                      \enspace | \enspace
            \eApp{\term}{\term}                             \enspace | \enspace
            \eTAbs{\tVar}{\kind}{\term}                     \enspace | \enspace
            \eTApp{\term}{\type}                            \) \\
          & & \(|\) & \(
            \eReify{\tVar}{\term}{\term}                    \enspace | \enspace
            \eReflect{\tVar}                                \) \\
          & & \(|\) & \(
            \ePure{\term}                                   \enspace | \enspace
            \eRun{\term}                                    \enspace | \enspace
            \eDo{\eVar}{\term}{\term}                       \) \\ \\
          types & \(\type, \effect\) & \(\Coloneqq\) & \(
            \tVar                                           \enspace | \enspace
            \tArrow{\type}{\type}                           \enspace | \enspace
            \tForAll{\tVar}{\kind}{\type}                   \enspace | \enspace
            \tComputation{\type}{\effect}                   \enspace | \enspace
            \tPure                                          \) \\ \\
          kinds & \(\kind\) & \(\Coloneqq\) & \(
            \kType                                          \enspace | \enspace
            \kEffect                                        \enspace | \enspace
            \kWitness{\type}                                \) \\ \\
          contexts & \(\context\) & \(\Coloneqq\) & \(
            \cEmpty                                         \enspace | \enspace
            \cEExtend{\context}{\eVar}{\type}               \enspace | \enspace
            \cTExtend{\context}{\tVar}{\kind}               \)
        \end{tabular}
      \end{center}

      \caption{Syntax}
      \label{fig:syntax}
    \end{figure}

  \subsection{Typing}

    \begin{figure}[H]
      \begin{center}
        \framebox{\(\hasType{\context}{\term}{\type}\)}
      \end{center}

      \medskip

      \begin{prooftree}
          \AxiomC{\(\apply{\context}{\eVar} = \type\)}
        \RightLabel{(\textsc{T-Var})}
        \UnaryInfC{\(\hasType{\context}{\eVar}{\type}\)}
      \end{prooftree}

      \begin{prooftree}
          \AxiomC{\(\hasType{\cEExtend{\context}{\eVar}{\type_1}}{\term}{\type_2}\)}
          \AxiomC{\(\hasKind{\context}{\type_1}{\kType}\)}
        \RightLabel{(\textsc{T-Abs})}
        \BinaryInfC{\(\hasType{\context}{\parens{\eAbs{\eVar}{\type_1}{\term}}}{\tArrow{\type_1}{\type_2}}\)}
      \end{prooftree}

      \begin{prooftree}
          \AxiomC{\(\hasType{\context}{\term_1}{\tArrow{\type_1}{\type_2}}\)}
          \AxiomC{\(\hasType{\context}{\term_2}{\type_1}\)}
        \RightLabel{(\textsc{T-App})}
        \BinaryInfC{\(\hasType{\context}{\eApp{\term_1}{\term_2}}{\type_2}\)}
      \end{prooftree}

      \begin{prooftree}
          \AxiomC{\(\hasType{\cTExtend{\context}{\tVar}{\kind}}{\term}{\type}\)}
          \AxiomC{\(\tVar \notin \dom{\context}\)}
        \RightLabel{(\textsc{T-TAbs})}
        \BinaryInfC{\(\hasType{\context}{\parens{\eTAbs{\tVar}{\kind}{\term}}}{\parens{\tForAll{\tVar}{\kind}{\type}}}\)}
      \end{prooftree}

      \begin{prooftree}
          \AxiomC{\(\hasType{\context}{\term}{\parens{\tForAll{\tVar}{\kind}{\type_1}}}\)}
          \AxiomC{\(\hasKind{\context}{\type_2}{\kind}\)}
        \RightLabel{(\textsc{T-TApp})}
        \BinaryInfC{\(\hasType{\context}{\eTApp{\term}{\type_2}}{\substitute{\type_1}{\tVar}{\type_2}}\)}
      \end{prooftree}

      \begin{prooftree}
          \AxiomC{\(\hasType{\context}{\term_1}{\type_1}\)}
          \AxiomC{\(\hasType{\cTExtend{\context}{\tVar}{\kWitness{\type_1}}}{\term_2}{\type_2}\)}
          \AxiomC{\(\hasKind{\context}{\type_2}{\kType}\)}
          \AxiomC{\(\tVar \notin \dom{\context}\)}
        \RightLabel{(\textsc{T-Reify})}
        \QuaternaryInfC{\(\hasType{\context}{\eReify{\tVar}{\term_1}{\term_2}}{\type_2}\)}
      \end{prooftree}

      \begin{prooftree}
          \AxiomC{\(\hasKind{\context}{\tVar}{\kWitness{\type}}\)}
        \RightLabel{(\textsc{T-Reflect})}
        \UnaryInfC{\(\hasType{\context}{\eReflect{\tVar}}{\type}\)}
      \end{prooftree}

      \begin{prooftree}
          \AxiomC{\(\hasType{\context}{\term_1}{\tComputation{\type_1}{\effect}}\)}
          \AxiomC{\(\hasType{\cEExtend{\context}{\eVar}{\type_1}}{\term_2}{\tComputation{\type_2}{\effect}}\)}
        \RightLabel{(\textsc{T-Do})}
        \BinaryInfC{\(\hasType{\context}{\eDo{\eVar}{\term_1}{\term_2}}{\tComputation{\type_2}{\effect}}\)}
      \end{prooftree}

      \begin{prooftree}
          \AxiomC{\(\hasType{\context}{\term}{\type}\)}
          \AxiomC{\(\hasKind{\context}{\effect}{\kEffect}\)}
        \RightLabel{(\textsc{T-Pure})}
        \BinaryInfC{\(\hasType{\context}{\ePure{\term}}{\tComputation{\type}{\effect}}\)}
      \end{prooftree}

      \begin{prooftree}
          \AxiomC{\(\hasType{\context}{\term}{\tComputation{\type}{\tPure}}\)}
        \RightLabel{(\textsc{T-Run})}
        \UnaryInfC{\(\hasType{\context}{\eRun{\term}}{\type}\)}
      \end{prooftree}

      \caption{Typing rules}
      \label{fig:typing}
    \end{figure}

    \begin{figure}[H]
      \begin{center}
        \framebox{\(\hasKind{\context}{\type}{\kind}\)}
      \end{center}

      \medskip

      \begin{prooftree}
          \AxiomC{\(\apply{\context}{\tVar} = \kind\)}
        \RightLabel{(\textsc{K-Var})}
        \UnaryInfC{\(\hasKind{\context}{\tVar}{\kind}\)}
      \end{prooftree}

      \begin{prooftree}
          \AxiomC{\(\hasKind{\context}{\type_1}{\kType}\)}
          \AxiomC{\(\hasKind{\context}{\type_2}{\kType}\)}
        \RightLabel{(\textsc{K-Arrow})}
        \BinaryInfC{\(\hasKind{\context}{\tArrow{\type_1}{\type_2}}{\kType}\)}
      \end{prooftree}

      \begin{prooftree}
          \AxiomC{\(\hasKind{\cTExtend{\context}{\tVar}{\kind}}{\type}{\kType}\)}
        \RightLabel{(\textsc{K-ForAll})}
        \UnaryInfC{\(\hasKind{\context}{\parens{\tForAll{\tVar}{\kind}{\type}}}{\kType}\)}
      \end{prooftree}

      \begin{prooftree}
          \AxiomC{\(\hasKind{\context}{\type}{\kType}\)}
          \AxiomC{\(\hasKind{\context}{\effect}{\kEffect}\)}
        \RightLabel{(\textsc{K-Computation})}
        \BinaryInfC{\(\hasKind{\context}{\tComputation{\type}{\effect}}{\kType}\)}
      \end{prooftree}

      \begin{prooftree}
          \AxiomC{}
        \RightLabel{(\textsc{K-Pure})}
        \UnaryInfC{\(\hasKind{\context}{\tPure}{\kEffect}\)}
      \end{prooftree}

      \caption{Kinding rules}
      \label{fig:kinding}
    \end{figure}

  \subsection{Semantics}

    Consider the program
    \[
      \eReify{\tVar_1}{\term}{\eAbs{\eVar}{\type_1}{\eTApp{\parens{\eTAbs{\tVar_2}{\kWitness{\type_2}}{\eReflect{\tVar_2}}}}{\tVar_1}}},
    \]
    which is well-typed when \(\hasType{\cEmpty}{\term}{\type_2}\). Intuition suggests the (weak) normal form \(\eAbs{\eVar}{\type_1}{\term}\). What might be the first step this program takes toward that normal form? One is tempted to eliminate the type application:
    \[
      \eReify{\tVar_1}{\term}{\eAbs{\eVar}{\type_1}{\eReflect{\tVar_1}}}.
    \]
    But that would be reduction under a lambda abstraction, which is ordinarily forbidden in lambda calculi intended for programming. Indeed, this modest calculus defies a straightforward operational semantics.

    However, the system admits a type-directed translation into call-by-value System F, shown in Figure~\ref{fig:semantics}.

    \begin{figure}[H]
      \ifnoacm
        \small
      \fi

      \begin{center}
        \framebox{\(\translatesTo{\context}{\term}{\term}\)}
        \framebox{\(\translatesTo{\context}{\type}{\type}\)}
        \framebox{\(\translatesTo{\context}{\type}{\term}\)}
      \end{center}

      \medskip

      \begin{center}
          \AxiomC{}
        \RightLabel{(\textsc{S-Var})}
        \UnaryInfC{\(\translatesTo{\context}{\eVar}{\eVar}\)}
        \DisplayProof
        \qquad
          \AxiomC{\(\translatesTo{\context}{\term_1}{\term_3}\)}
          \AxiomC{\(\translatesTo{\context}{\term_2}{\term_4}\)}
        \RightLabel{(\textsc{S-App})}
        \BinaryInfC{\(\translatesTo{\context}{\eApp{\term_1}{\term_2}}{\eApp{\term_3}{\term_4}}\)}
        \DisplayProof
      \end{center}

      \begin{center}
          \AxiomC{\(\translatesTo{\context}{\type_1}{\type_2}\)}
          \AxiomC{\(\translatesTo{\context}{\term_1}{\term_2}\)}
        \RightLabel{(\textsc{S-Abs1})}
        \BinaryInfC{\(\translatesTo{\context}{\eAbs{\eVar}{\type_1}{\term_1}}{\eAbs{\eVar}{\type_2}{\term_2}}\)}
        \DisplayProof
        \qquad
          \AxiomC{\(\translatesTo{\context}{\type}{\term_3}\)}
          \AxiomC{\(\translatesTo{\context}{\term_1}{\term_2}\)}
        \RightLabel{(\textsc{S-Abs2})}
        \BinaryInfC{\(\translatesTo{\context}{\eAbs{\eVar}{\type}{\term_1}}{\eAbs{\eVar}{\tVoid}{\term_2}}\)}
        \DisplayProof
      \end{center}

      \begin{center}
          \AxiomC{\(\translatesTo{\context}{\term_1}{\term_2}\)}
        \RightLabel{(\textsc{S-TAbs1})}
        \UnaryInfC{\(\translatesTo{\context}{\eTAbs{\tVar}{\kType}{\term_1}}{\eTAbsNoKind{\tVar}{\term_2}}\)}
        \DisplayProof
        \qquad
          \AxiomC{\(\translatesTo{\context}{\term_1}{\term_2}\)}
        \RightLabel{(\textsc{S-TAbs2})}
        \UnaryInfC{\(\translatesTo{\context}{\eTAbs{\tVar}{\kEffect}{\term_1}}{\eTAbsNoKind{\tVar}{\term_2}}\)}
        \DisplayProof
      \end{center}

      \begin{center}
          \AxiomC{\(\translatesTo{\context}{\type_1}{\type_2}\)}
          \AxiomC{\(\translatesTo{\context}{\term_1}{\term_2}\)}
        \RightLabel{(\textsc{S-TAbs3})}
        \BinaryInfC{\(\translatesTo{\context}{\eTAbs{\tVar}{\kWitness{\type_1}}{\term_1}}{\eAbs{\eVar_{\tVar}}{\type_2}{\term_2}}\)}
        \DisplayProof
        \qquad
          \AxiomC{\(\translatesTo{\context}{\type}{\term_3}\)}
          \AxiomC{\(\translatesTo{\context}{\term_1}{\term_2}\)}
        \RightLabel{(\textsc{S-TAbs4})}
        \BinaryInfC{\(\translatesTo{\context}{\eTAbs{\tVar}{\kWitness{\type}}{\term_1}}{\eAbs{\eVar_{\tVar}}{\tVoid}{\term_2}}\)}
        \DisplayProof
      \end{center}

      \begin{center}
          \AxiomC{\(\translatesTo{\context}{\term_1}{\term_2}\)}
          \AxiomC{\(\translatesTo{\context}{\type_1}{\type_2}\)}
        \RightLabel{(\textsc{S-TApp1})}
        \BinaryInfC{\(\translatesTo{\context}{\eTApp{\term_1}{\type_1}}{\eTApp{\term_2}{\type_2}}\)}
        \DisplayProof
        \qquad
          \AxiomC{\(\translatesTo{\context}{\term_1}{\term_2}\)}
          \AxiomC{\(\translatesTo{\context}{\type}{\term_3}\)}
        \RightLabel{(\textsc{S-TApp2})}
        \BinaryInfC{\(\translatesTo{\context}{\eTApp{\term_1}{\type}}{\eApp{\term_2}{\term_3}}\)}
        \DisplayProof
      \end{center}

      \begin{center}
          \AxiomC{\(\translatesTo{\context}{\term_1}{\term_3}\)}
          \AxiomC{\(\translatesTo{\context}{\term_2}{\term_4}\)}
        \RightLabel{(\textsc{S-Reify})}
        \BinaryInfC{\(\translatesTo{\context}{\eReify{\tVar}{\term_1}{\term_2}}{\substitute{\term_4}{\eVar_{\tVar}}{\term_3}}\)}
        \DisplayProof
        \qquad
          \AxiomC{}
        \RightLabel{(\textsc{S-Reflect})}
        \UnaryInfC{\(\translatesTo{\context}{\eReflect{\tVar}}{\eVar_{\tVar}}\)}
        \DisplayProof
      \end{center}

      \begin{center}
          \AxiomC{\(\translatesTo{\context}{\term_1}{\term_2}\)}
        \RightLabel{(\textsc{S-Pure})}
        \UnaryInfC{\(\translatesTo{\context}{\ePure{\term_1}}{\eAbs{\wildcard}{\tUnit}{\term_2}}\)}
        \DisplayProof
        \qquad
          \AxiomC{\(\translatesTo{\context}{\term_1}{\term_2}\)}
        \RightLabel{(\textsc{S-Run})}
        \UnaryInfC{\(\translatesTo{\context}{\eRun{\term_1}}{\eApp{\term_2}{\eUnit}}\)}
        \DisplayProof
      \end{center}

      \begin{prooftree}
          \AxiomC{\(\hasType{\context}{\term_1}{\type_1}\)}
          \AxiomC{\(\translatesTo{\context}{\type_1}{\type_2}\)}
          \AxiomC{\(\translatesTo{\context}{\term_1}{\term_3}\)}
          \AxiomC{\(\translatesTo{\context}{\term_2}{\term_4}\)}
        \RightLabel{(\textsc{S-Do})}
        \QuaternaryInfC{\(\translatesTo{\context}{\eDo{\eVar}{\term_1}{\term_2}}{\eApp{\parens{\eAbs{\eVar}{\type_2}{\term_4}}}{\term_3}}\)}
      \end{prooftree}

      \begin{center}
          \AxiomC{\(\hasKind{\context}{\tVar}{\kType}\)}
        \RightLabel{(\textsc{S-TVar1})}
        \UnaryInfC{\(\translatesTo{\context}{\tVar}{\tVar}\)}
        \DisplayProof
        \qquad
          \AxiomC{\(\hasKind{\context}{\tVar}{\kWitness{\type}}\)}
        \RightLabel{(\textsc{S-TVar2})}
        \UnaryInfC{\(\translatesTo{\context}{\tVar}{\eVar_{\tVar}}\)}
        \DisplayProof
      \end{center}

      \begin{center}
          \AxiomC{\(\translatesTo{\context}{\type_1}{\type_3}\)}
          \AxiomC{\(\translatesTo{\context}{\type_2}{\type_4}\)}
        \RightLabel{(\textsc{S-Arrow1})}
        \BinaryInfC{\(\translatesTo{\context}{\tArrow{\type_1}{\type_2}}{\tArrow{\type_3}{\type_4}}\)}
        \DisplayProof
        \qquad
          \AxiomC{\(\translatesTo{\context}{\type_1}{\type_3}\)}
          \AxiomC{\(\translatesTo{\context}{\type_2}{\term}\)}
        \RightLabel{(\textsc{S-Arrow2})}
        \BinaryInfC{\(\translatesTo{\context}{\tArrow{\type_1}{\type_2}}{\tArrow{\type_3}{\tVoid}}\)}
        \DisplayProof
      \end{center}

      \begin{center}
          \AxiomC{\(\translatesTo{\context}{\type_1}{\term}\)}
          \AxiomC{\(\translatesTo{\context}{\type_2}{\type_3}\)}
        \RightLabel{(\textsc{S-Arrow3})}
        \BinaryInfC{\(\translatesTo{\context}{\tArrow{\type_1}{\type_2}}{\tArrow{\tVoid}{\type_3}}\)}
        \DisplayProof
        \qquad
          \AxiomC{\(\translatesTo{\context}{\type_1}{\term_1}\)}
          \AxiomC{\(\translatesTo{\context}{\type_2}{\term_2}\)}
        \RightLabel{(\textsc{S-Arrow4})}
        \BinaryInfC{\(\translatesTo{\context}{\tArrow{\type_1}{\type_2}}{\tArrow{\tVoid}{\tVoid}}\)}
        \DisplayProof
      \end{center}

      \begin{center}
          \AxiomC{\(\translatesTo{\context}{\type_1}{\type_2}\)}
        \RightLabel{(\textsc{S-ForAll1})}
        \UnaryInfC{\(\translatesTo{\context}{\tForAll{\tVar}{\kType}{\type_1}}{\tForAllNoKind{\tVar}{\type_2}}\)}
        \DisplayProof
        \qquad
          \AxiomC{\(\translatesTo{\context}{\type}{\term}\)}
        \RightLabel{(\textsc{S-ForAll2})}
        \UnaryInfC{\(\translatesTo{\context}{\tForAll{\tVar}{\kType}{\type}}{\tForAllNoKind{\tVar}{\tVoid}}\)}
        \DisplayProof
      \end{center}

      \begin{center}
          \AxiomC{\(\translatesTo{\context}{\type_1}{\type_3}\)}
          \AxiomC{\(\translatesTo{\context}{\type_2}{\type_4}\)}
        \RightLabel{(\textsc{S-ForAll3})}
        \BinaryInfC{\(\translatesTo{\context}{\tForAll{\tVar}{\kWitness{\type_1}}{\type_2}}{\tArrow{\type_3}{\type_4}}\)}
        \DisplayProof
        \qquad
          \AxiomC{\(\translatesTo{\context}{\type_1}{\type_3}\)}
          \AxiomC{\(\translatesTo{\context}{\type_2}{\term}\)}
        \RightLabel{(\textsc{S-ForAll4})}
        \BinaryInfC{\(\translatesTo{\context}{\tForAll{\tVar}{\kWitness{\type_1}}{\type_2}}{\tArrow{\type_3}{\tVoid}}\)}
        \DisplayProof
      \end{center}

      \begin{center}
          \AxiomC{\(\translatesTo{\context}{\type_1}{\term}\)}
          \AxiomC{\(\translatesTo{\context}{\type_2}{\type_3}\)}
        \RightLabel{(\textsc{S-ForAll5})}
        \BinaryInfC{\(\translatesTo{\context}{\tForAll{\tVar}{\kWitness{\type_1}}{\type_2}}{\tArrow{\tVoid}{\type_3}}\)}
        \DisplayProof
        \qquad
          \AxiomC{\(\translatesTo{\context}{\type_1}{\term_1}\)}
          \AxiomC{\(\translatesTo{\context}{\type_2}{\term_2}\)}
        \RightLabel{(\textsc{S-ForAll6})}
        \BinaryInfC{\(\translatesTo{\context}{\tForAll{\tVar}{\kWitness{\type_1}}{\type_2}}{\tArrow{\tVoid}{\tVoid}}\)}
        \DisplayProof
      \end{center}

      \begin{center}
          \AxiomC{\(\translatesTo{\context}{\type_1}{\type_2}\)}
        \RightLabel{(\textsc{S-Computation})}
        \UnaryInfC{\(\translatesTo{\context}{\tComputation{\type_1}{\effect}}{\tArrow{\tUnit}{\type_2}}\)}
        \DisplayProof
          \AxiomC{\(\hasKind{\context}{\effect}{\kEffect}\)}
        \RightLabel{(\textsc{S-Effect})}
        \UnaryInfC{\(\translatesTo{\context}{\effect}{\tUnit}\)}
        \DisplayProof
      \end{center}

      \caption{Elaboration into System F}
      \label{fig:semantics}
    \end{figure}


  \ifnoacm
    \section{Acknowledgements}
      This section is temporarily omitted.

  \fi

  \ifacm
% For some reason, LaTeX throws an error if this is indented.
\begin{acks}
  This section is temporarily omitted.

\end{acks}
  \fi

  \ifacm
    \bibliographystyle{ACM-Reference-Format}
  \else
    \bibliographystyle{plainnat}
  \fi

  \bibliography{references}
\end{document}
