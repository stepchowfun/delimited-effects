%%%%%%%%%%%%%%%%%%%%%%%%%%%%%%%%%%
% Document settings and packages %
%%%%%%%%%%%%%%%%%%%%%%%%%%%%%%%%%%

\documentclass[12pt]{article}

\usepackage[T1]{fontenc} % The font encoding
\usepackage[framemethod=tikz]{mdframed} % For figures (`tikz` allows us to have a transparent background)
\usepackage[margin=1in]{geometry} % For setting the margin
\usepackage{amssymb} % For \varnothing
\usepackage{bussproofs} % For inference rules
\usepackage{float} % For the [H] option in \begin{figure}[H]. NOTE: Stop using this when we have more content
\usepackage{inconsolata} % Monospace font
\usepackage{listings} % For source code
\usepackage{mathtools} % For \Coloneqq
\usepackage{stackengine} % For stacking axioms
\usepackage{stmaryrd} % For \llbracket and \rrbracket
\usepackage{xcolor}

% Add a "Draft" watermark to the background.
\usepackage{draftwatermark}
\SetWatermarkText{\textsc{Draft}}
\SetWatermarkScale{3}
\SetWatermarkLightness{0.975}

\title{Delimited effects}
\date{}

%%%%%%%%%%%%%%%%%%%%%%%%%%
% Source code formatting %
%%%%%%%%%%%%%%%%%%%%%%%%%%

\definecolor{purple}{RGB}{127, 0, 181}

\lstdefinelanguage{gram}{morekeywords={effect, handle}}

\lstset{
  aboveskip=\bigskipamount,
  basicstyle=\ttfamily,
  belowskip=\bigskipamount,
  keepspaces=true,
  keywordstyle=\bfseries\color{purple},
  language=gram,
}

%%%%%%%%%%
% Macros %
%%%%%%%%%%

% Theorem styles
\newtheorem{definition}{Definition}
\newtheorem{theorem}{Theorem}

% Misc
\newcommand\anno[2]{#1 : #2}
\newcommand\apply[2]{#1\parens{#2}}
\newcommand\parens[1]{\left( #1 \right)}
\newcommand\fv[1]{\apply{\text{fv}}{#1}}
\newcommand\ftv[1]{\apply{\text{ftv}}{#1}}
\newcommand\eff[1]{\apply{\text{eff}}{#1}}
\newcommand\dom[1]{\apply{\text{dom}}{#1}}
\newcommand\substitute[3]{#1 \left[ #3 / #2 \right]}
\makeatletter
\renewcommand{\boxed}[1]{\text{\kern 0.1em\fboxsep=.2em\fbox{\m@th$\displaystyle#1$}\kern 0.1em}}
\makeatother

% Terms
\newcommand\term{t}
\newcommand\eVar{x}
\newcommand\eAbs[2]{\lambda #1 \; . \; #2}
\newcommand\eApp[2]{#1 \; #2}
\newcommand\eTAbs[2]{\Lambda #1 \; . \; #2}
\newcommand\eTApp[2]{#1 \; #2}
\newcommand\eHandle[3]{\textbf{handle} \; #1 \; \textbf{with} \; #2 \; \textbf{in} \; #3}
\newcommand\eEffect[5]{\textbf{effect} \; \anno{#1}{#2} \; \textbf{with} \; \tEmbellished{#3}{#4} \; \textbf{in} \; #5}
\newcommand\eAnno[2]{\anno{#1}{#2}}

% Types
\newcommand\type{\tau}
\newcommand\tVar{\alpha}
\newcommand\tArrow[4]{\tEmbellished{#1}{#2} \rightarrow \tEmbellished{#3}{#4}}
\newcommand\tForall[3]{\forall #1 \; . \; \tEmbellished{#2}{#3}}
\newcommand\tEmbellished[2]{{#1}^{\textcolor{violet}{#2}}}

% Rows
\newcommand\row{\varepsilon}
\newcommand\rEmpty{\varnothing}
\newcommand\rSingleton[1]{\left\{ #1 \right\}}
\newcommand\rUnion[2]{#1 \cup #2}

% Type contexts
\newcommand\context{\Gamma}
\newcommand\cEmpty{\varnothing}
\newcommand\cTExtend[4]{#1, \anno{#2}{\tEmbellished{#3}{#4}}}
\newcommand\cKExtend[2]{#1, #2}

% Effect map
\newcommand\effectMap{\Sigma}
\newcommand\emEmpty{\varnothing}
\newcommand\emExtend[4]{#1, #2 \mapsto \tEmbellished{#3}{#4}}

% Judgments
\newcommand\subrowSym{\subseteq}
\newcommand\nSubrowSym{\nsubseteq}
\newcommand\subrow[2]{#1 \subrowSym #2}
\newcommand\nSubrow[2]{#1 \nSubrowSym #2}
\newcommand\checkType[5]{#1 ; #2 \vdash #3 \Downarrow \tEmbellished{#4}{#5}}
\newcommand\inferType[5]{#1 ; #2 \vdash #3 \Uparrow \tEmbellished{#4}{#5}}

%%%%%%%%%%%%%%%%%%%%%
% The specification %
%%%%%%%%%%%%%%%%%%%%%

\begin{document}
  \maketitle

  \begin{abstract}
    We introduce delimited effects, a modest type and effect system capable of modeling a diverse collection of programming language features including algebraic effects, type classes, and dynamic scoping.
  \end{abstract}

  \section{Introduction}

    \iffalse
      \begin{lstlisting}[gobble=4]
        effect IO
          getLine   : String ! IO
          printLine : String -> () ! IO
      \end{lstlisting}

      \begin{lstlisting}[gobble=4]
        effect Monoid a
          mempty  : a ! Monoid a
          mappend : a -> a -> a ! Monoid a
      \end{lstlisting}
    \fi

  \section{Overview}

    \subsection{Syntax}

      \begin{figure}[H]
        \begin{mdframed}[backgroundcolor=none]
          \begin{center}
            \begin{tabular}{l l l}
              $\term \Coloneqq$ & & terms: \\
              & $\eVar$ & variable \\
              & $\eAbs{\eVar}{\term}$ & abstraction \\
              & $\eApp{\term}{\term}$ & application \\
              & $\eTAbs{\tVar}{\term}$ & type abstraction \\
              & $\eTApp{\term}{\type}$ & type application \\
              & $\eEffect{\eVar}{\tVar}{\type}{\row}{\term}$ & effect definition \\
              & $\eHandle{\tVar}{\term}{\term}$ & effect handler \\
              & $\eAnno{\term}{\type}$ & type annotation \\
              \\
              $\type \Coloneqq$ & & types: \\
              & $\tVar$ & type variable \\
              & $\tArrow{\type}{\row}{\type}{\row}$ & arrow type \\
              & $\tForall{\tVar}{\type}{\row}$ & universal type \\
              \\
              $\row \Coloneqq$ & & rows: \\
              & $\rEmpty$ & empty row \\
              & $\rSingleton{\tVar}$ & singleton row \\
              & $\rUnion{\row}{\row}$ & row union \\
              \\
              $\context \Coloneqq$ & & contexts: \\
              & $\cEmpty$ & empty context \\
              & $\cTExtend{\context}{\eVar}{\type}{\row}$ & variable binding \\
              & $\cKExtend{\context}{\tVar}$ & type variable binding \\
              \\
              $\effectMap \Coloneqq$ & & effect maps: \\
              & $\emEmpty$ & empty effect map \\
              & $\emExtend{\effectMap}{\tVar}{\type}{\row}$ & effect binding \\
            \end{tabular}
          \end{center}

          \caption{Syntax}\label{fig:syntax}
        \end{mdframed}
      \end{figure}

    \subsection{Typing}

      \begin{figure}[H]
        \begin{mdframed}[backgroundcolor=none]
          \begin{center}
            \framebox{
              $\checkType{\context}{\effectMap}{\term}{\type}{\row}$
              \qquad
              $\inferType{\context}{\effectMap}{\term}{\type}{\row}$
            }
          \end{center}

          \medskip

          \begin{prooftree}
              \AxiomC{$\tEmbellished{\type}{\row} = \apply{\context}{\eVar}$}
            \RightLabel{(\textsc{T-Var})}
            \UnaryInfC{$\inferType{\context}{\effectMap}{\eVar}{\type}{\row}$}
          \end{prooftree}

          \begin{prooftree}
              \AxiomC{$\checkType{\cTExtend{\context}{\eVar}{\type_2}{\row_2}}{\effectMap}{\term}{\type_1}{\row_1}$}
            \RightLabel{(\textsc{T-Abs})}
            \UnaryInfC{$\checkType{\context}{\effectMap}{\eAbs{\eVar}{\term}}{\parens{\tArrow{\type_2}{\row_2}{\type_1}{\row_1}}}{\row_3}$}
          \end{prooftree}

          \begin{prooftree}
              \AxiomC{$\inferType{\context}{\effectMap}{\term_1}{\parens{\tArrow{\type_2}{\row_2}{\type_1}{\row_1}}}{\row_3}$}
              \AxiomC{$\checkType{\context}{\effectMap}{\term_2}{\type_2}{\rUnion{\row_2}{\row_3}}$}
              \AxiomC{$\subrow{\row_1}{\row_3}$}
            \RightLabel{(\textsc{T-App})}
            \TrinaryInfC{$\inferType{\context}{\effectMap}{\eApp{\term_1}{\term_2}}{\type_1}{\row_3}$}
          \end{prooftree}

          \begin{prooftree}
              \AxiomC{$\tVar \notin \dom{\context}$}
              \AxiomC{$\checkType{\cKExtend{\context}{\tVar}}{\effectMap}{\term}{\type}{\row_1}$}
            \RightLabel{(\textsc{T-TAbs})}
            \BinaryInfC{$\checkType{\context}{\effectMap}{\eTAbs{\tVar}{\term}}{\parens{\tForall{\tVar}{\type}{\row_1}}}{\row_2}$}
          \end{prooftree}

          \begin{prooftree}
              \AxiomC{$\inferType{\context}{\effectMap}{\term}{\parens{\tForall{\tVar}{\type_1}{\row_1}}}{\row_2}$}
              \AxiomC{$\subrow{\row_1}{\row_2}$}
            \RightLabel{(\textsc{T-TApp})}
            \BinaryInfC{$\inferType{\context}{\effectMap}{\eTApp{\term}{\type_2}}{\substitute{\type_1}{\tVar}{\type_2}}{\row_2}$}
          \end{prooftree}

          \caption{Type inference (part 1)}\label{fig:typing_1}
        \end{mdframed}
      \end{figure}

      \begin{figure}[H]
        \begin{mdframed}[backgroundcolor=none]
          \begin{center}
            \framebox{
              $\checkType{\context}{\effectMap}{\term}{\type}{\row}$
              \qquad
              $\inferType{\context}{\effectMap}{\term}{\type}{\row}$
            }
          \end{center}

          \medskip

          \begin{prooftree}
              \AxiomC{$\inferType{\cTExtend{\context}{\eVar}{\type_1}{\row_1}}{\emExtend{\effectMap}{\tVar}{\type_1}{\row_1}}{\term}{\type_2}{\row}$}
            \RightLabel{(\textsc{T-Effect1})}
            \UnaryInfC{$\inferType{\context}{\effectMap}{\eEffect{\eVar}{\tVar}{\type_1}{\row_1}{\term}}{\type_2}{\row_2}$}
          \end{prooftree}

          \begin{prooftree}
              \AxiomC{$\checkType{\cTExtend{\context}{\eVar}{\type_1}{\row_1}}{\emExtend{\effectMap}{\tVar}{\type_1}{\row_1}}{\term}{\type_2}{\row}$}
            \RightLabel{(\textsc{T-Effect2})}
            \UnaryInfC{$\checkType{\context}{\effectMap}{\eEffect{\eVar}{\tVar}{\type_1}{\row_1}{\term}}{\type_2}{\row_2}$}
          \end{prooftree}

          \begin{prooftree}
              \AxiomC{\Shortstack[c]{
                {$\tEmbellished{\type_2}{\row_2} = \apply{\effectMap}{\tVar}$}
                {$\type_3 = \substitute{\type_2}{\rSingleton{\tVar}}{\row_1}$}
                {$\row_3 = \substitute{\row_2}{\rSingleton{\tVar}}{\row_1}$}
                {$\checkType{\context}{\effectMap}{\term_1}{\type_3}{\row_3}$}
                {$\inferType{\context}{\effectMap}{\term_2}{\type_1}{\rUnion{\row_1}{\rSingleton{\tVar}}}$}
              }}
            \RightLabel{(\textsc{T-Handle1})}
            \UnaryInfC{$\inferType{\context}{\effectMap}{\eHandle{\tVar}{\term_1}{\term_2}}{\type_1}{\row_1}$}
          \end{prooftree}

          \begin{prooftree}
              \AxiomC{\Shortstack[c]{
                {$\tEmbellished{\type_2}{\row_2} = \apply{\effectMap}{\tVar}$}
                {$\type_3 = \substitute{\type_2}{\rSingleton{\tVar}}{\row_1}$}
                {$\row_3 = \substitute{\row_2}{\rSingleton{\tVar}}{\row_1}$}
                {$\checkType{\context}{\effectMap}{\term_1}{\type_3}{\row_3}$}
                {$\checkType{\context}{\effectMap}{\term_2}{\type_1}{\rUnion{\row_1}{\rSingleton{\tVar}}}$}
              }}
            \RightLabel{(\textsc{T-Handle2})}
            \UnaryInfC{$\checkType{\context}{\effectMap}{\eHandle{\tVar}{\term_1}{\term_2}}{\type_1}{\row_1}$}
          \end{prooftree}

          \begin{prooftree}
              \AxiomC{$\checkType{\context}{\effectMap}{\term}{\type}{\row}$}
            \RightLabel{(\textsc{T-Anno})}
            \UnaryInfC{$\inferType{\context}{\effectMap}{\eAnno{\term}{\type}}{\type}{\row}$}
          \end{prooftree}

          \begin{prooftree}
              \AxiomC{$\inferType{\context}{\effectMap}{\term}{\type_2}{\row}$}
              \AxiomC{$\type_1 = \type_2$}
            \RightLabel{(\textsc{T-Sub})}
            \BinaryInfC{$\checkType{\context}{\effectMap}{\term}{\type_1}{\row}$}
          \end{prooftree}

          \caption{Type inference (part 2)}\label{fig:typing_2}
        \end{mdframed}
      \end{figure}
\end{document}
