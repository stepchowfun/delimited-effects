%%%%%%%%%%%%%%%%%%%%%%%%%%%%%%%%%%
% Document settings and packages %
%%%%%%%%%%%%%%%%%%%%%%%%%%%%%%%%%%

\documentclass[12pt]{article}

\usepackage[T1]{fontenc} % The font encoding
\usepackage[framemethod=tikz]{mdframed} % For figures (`tikz` allows us to have a transparent background)
\usepackage[margin=1in]{geometry} % For setting the margin
\usepackage{amssymb} % For \varnothing
\usepackage{bussproofs} % For inference rules
\usepackage{float} % For the [H] option in \begin{figure}[H]. NOTE: Stop using this when we have more content
\usepackage{inconsolata} % Monospace font
\usepackage{listings} % For source code
\usepackage{mathtools} % For \Coloneqq
\usepackage{stackengine} % For stacking axioms
\usepackage{stmaryrd} % For \llbracket and \rrbracket
\usepackage{xcolor}

% Add a "Draft" watermark to the background.
\usepackage{draftwatermark}
\SetWatermarkText{\textsc{Draft}}
\SetWatermarkScale{3}
\SetWatermarkLightness{0.975}

\title{Delimited effects}
\date{}

%%%%%%%%%%%%%%%%%%%%%%%%%%
% Source code formatting %
%%%%%%%%%%%%%%%%%%%%%%%%%%

\definecolor{purple}{RGB}{127, 0, 181}

\lstdefinelanguage{gram}{morekeywords={effect, provide}}

\lstset{
  aboveskip=\bigskipamount,
  basicstyle=\ttfamily,
  belowskip=\bigskipamount,
  keepspaces=true,
  keywordstyle=\bfseries\color{purple},
  language=gram,
}

%%%%%%%%%%
% Macros %
%%%%%%%%%%

% Theorem styles
\newtheorem{definition}{Definition}
\newtheorem{theorem}{Theorem}

% Misc
\newcommand\anno[2]{#1 : #2}
\newcommand\apply[2]{#1\parens{#2}}
\newcommand\parens[1]{\left( #1 \right)}
\newcommand\substitute[3]{#1 \left[ #3 / #2 \right]}

% Terms
\newcommand\term{t}
\newcommand\eunit{()}
\newcommand\evar{x}
\newcommand\eabs[2]{\lambda #1 \; . \; #2}
\newcommand\eapp[2]{#1 \; #2}
\newcommand\eprovide[4]{\textbf{provide} \; #1 / #2 \; \textbf{with} \; #3 \; \textbf{in} \; #4}

% Types
\newcommand\tembellished[2]{{#1}^{\textcolor{violet}{#2}}}
\newcommand\type{\tau}
\newcommand\tunit{1}
\newcommand\tarrow[2]{#1 \rightarrow #2}

% Rows
\newcommand\row{\varepsilon}
\newcommand\rempty{\varnothing_{\row}}
\newcommand\rsingleton[1]{\left\{ #1 \right\}}
\newcommand\runion[2]{#1 \cup #2}
\newcommand\rdiff[2]{#1 \setminus #2}

% Type contexts
\newcommand\context{\Gamma}
\newcommand\cempty{\varnothing_{\context}}
\newcommand\cextend[2]{#1, #2}

% Effect map
\newcommand\effect{e}
\newcommand\effectmap{E}
\newcommand\emempty{\varnothing_{\effectmap}}
\newcommand\emmap[2]{#1 \mapsto #2}
\newcommand\emextend[2]{#1, #2}

% Judgments
\newcommand\hastype[3]{#1 \vdash \anno{#2}{#3}}
\newcommand\ewellformed[1]{#1 \; \text{well-formed effect}}

%%%%%%%%%%%%%%%%%%%%%
% The specification %
%%%%%%%%%%%%%%%%%%%%%

\begin{document}
  \maketitle

  \begin{abstract}
    We introduce delimited effects, a modest type and effect system capable of modeling a diverse collection of programming language features including algebraic effects, type classes, and dynamic scoping.
  \end{abstract}

  \section{Introduction}

    \iffalse
      \begin{lstlisting}[gobble=4]
        effect IO
          getLine   : String ! IO
          printLine : String -> () ! IO
      \end{lstlisting}

      \begin{lstlisting}[gobble=4]
        effect Monoid a
          mempty  : a ! Monoid a
          mappend : a -> a -> a ! Monoid a
      \end{lstlisting}
    \fi

  \section{Overview}

    \subsection{Syntax and semantics}

      \begin{figure}[H]
        \begin{mdframed}[backgroundcolor=none]
          \begin{center}
            \begin{tabular}{l l l}
              $\term \Coloneqq $ & & terms: \\
              & $\eunit$ & unit \\
              & $\evar$ & variable \\
              & $\eabs{\anno{\evar}{\type}}{\term}$ & abstraction \\
              & $\eapp{\term}{\term}$ & application \\
              & $\eprovide{\effect}{\overline{\effect_i}}{\term}{\term}$ & effect definition \\
              \\
              $\type \Coloneqq$ & & types: \\
              & $\tunit$ & unit type \\
              & $\tarrow{\type}{\tembellished{\type}{\row}}$ & arrow type \\
              \\
              $\row \Coloneqq$ & & rows: \\
              & $\rempty$ & empty row \\
              & $\rsingleton{\effect}$ & singleton row \\
              & $\runion{\row}{\row}$ & row union \\
              & $\rdiff{\row}{\row}$ & row difference \\
              \\
              $\context \Coloneqq$ & & contexts: \\
              & $\cempty$ & empty context \\
              & $\cextend{\context}{\anno{\evar}{\tembellished{\type}{\row}}}$ & variable binding \\
              \\
              $\effectmap \Coloneqq$ & & effect map: \\
              & $\emempty$ & empty effect map \\
              & $\emextend{\effectmap}{\emmap{\effect}{\anno{\evar}{\tembellished{\type}{\row}}}}$ & effect binding \\
            \end{tabular}
          \end{center}

          \caption{Syntax}\label{fig:syntax}
        \end{mdframed}
      \end{figure}

      \begin{figure}[H]
        \begin{mdframed}[backgroundcolor=none]
          \begin{center}
            \framebox{$\hastype{\context}{\term}{\tembellished{\type}{\row}}$}
          \end{center}

          \medskip

          \begin{prooftree}
              \AxiomC{}
            \RightLabel{(\textsc{T-Unit})}
            \UnaryInfC{$\hastype{\context}{\eunit}{\tembellished{\tunit}{\rempty}}$}
          \end{prooftree}

          \begin{prooftree}
              \AxiomC{$\apply{\context}{\evar} = \tembellished{\type}{\row}$}
            \RightLabel{(\textsc{T-Variable})}
            \UnaryInfC{$\hastype{\context}{\evar}{\tembellished{\type}{\row}}$}
          \end{prooftree}

          \begin{prooftree}
              \AxiomC{$\hastype{\cextend{\context}{\anno{\evar}{\tembellished{\type_1}{\rempty}}}}{\term}{\tembellished{\type_2}{\row_2}}$}
            \RightLabel{(\textsc{T-Abstraction})}
            \UnaryInfC{$\hastype{\context}{\parens{\eabs{\anno{\evar}{\type_1}}{\term}}}{\tembellished{\parens{\tarrow{\type_1}{\tembellished{\type_2}{\row_2}}}}{\rempty}}$}
          \end{prooftree}

          \begin{prooftree}
              \AxiomC{$\hastype{\context}{\term_1}{\tembellished{\type_1}{\row_1}}$}
              \AxiomC{$\hastype{\context}{\term_2}{\tembellished{\parens{\tarrow{\type_1}{\tembellished{\type_2}{\row_2}{}}}}{\row_3}}$}
            \RightLabel{(\textsc{T-Application})}
            \BinaryInfC{$\hastype{\context}{\eapp{\term_2}{\term_1}}{\tembellished{\type_2}{\runion{\runion{\row_1}{\row_2}}{\row_3}}}$}
          \end{prooftree}

          \begin{prooftree}
              \AxiomC{$\hastype{\context}{\term_1}{\tembellished{\type_1}{\row_1}}$}
              \AxiomC{$\hastype{\context}{\term_2}{\tembellished{\type_2}{\row_2}}$}
              \AxiomC{$\apply{\effectmap}{\effect} = \anno{\evar}{\tembellished{\substitute{\type_1}{\effect_i}{\effect}}{\runion{\substitute{\row_1}{\effect_i}{\effect}}{\row_3}}}$}
            \RightLabel{(\textsc{T-Provide})}
            \TrinaryInfC{$\hastype{\context}{\eprovide{\effect}{\overline{\effect_i}}{\term_1}{\term_2}}{\tembellished{\type_2}{\runion{\parens{\rdiff{\row_2}{\rsingleton{\effect}}}}{\effect_i}}}$}
          \end{prooftree}

          \caption{Typing rules}\label{fig:typing_rules}
        \end{mdframed}
      \end{figure}

      \begin{figure}[H]
        \begin{mdframed}[backgroundcolor=none]
          \begin{center}
            \framebox{$\ewellformed{\emmap{\effect}{\tembellished{\type}{\row}}}$}
          \end{center}

          \medskip

          \begin{prooftree}
              \AxiomC{$\effect \in \row$}
            \RightLabel{(\textsc{WFE-Unit})}
            \UnaryInfC{$\ewellformed{\emmap{\effect}{\tembellished{\tunit}{\row}}}$}
          \end{prooftree}

          \begin{prooftree}
              \AxiomC{$\effect \in \row_2$}
            \RightLabel{(\textsc{WFE-Arrow1})}
            \UnaryInfC{$\ewellformed{\emmap{\effect}{\tembellished{\parens{\tarrow{\type_1}{\tembellished{\type_2}{\row_1}}}}{\row_2}}}$}
          \end{prooftree}

          \begin{prooftree}
              \AxiomC{$\ewellformed{\emmap{\effect}{\tembellished{\type_2}{\row_1}}}$}
            \RightLabel{(\textsc{WFE-Arrow2})}
            \UnaryInfC{$\ewellformed{\emmap{\effect}{\tembellished{\parens{\tarrow{\type_1}{\tembellished{\type_2}{\row_1}}}}{\row_2}}}$}
          \end{prooftree}

          \caption{Effect well-formedness}\label{fig:effect_well_formedness}
        \end{mdframed}
      \end{figure}

\end{document}
