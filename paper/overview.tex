\section{Overview}

  \subsection{Reflection and reification}

    Here's a simple example of reification and reflection.
    \[ \eReify{4}{\tVar}{\parens{3 + \eReflect{\tVar}}} \]

  \subsection{Type classes with local instances}

    We can represent a type class as an ordinary data type.
    \[ \eData{\tMonoid{\tVar}}{\eMonoid{\tVar}{\parens{\tArrow{\tVar}{\tVar}}}} \]
    For convenience, we define \emph{methods} for the class. These are functions that reflect a reified instance back into a term and extract the appropriate fields.
    \[ \eIdentity \Coloneqq \eTAbs{\tVar_1}{\kType}{\eTAbs{\tVar_2}{\kWitness{\tMonoid{\tVar_1}}}{\eLet{\eMonoid{\eVar}{\wildcard}}{\eReflect{\tVar_2}}{\eVar}}} \]
    \[ \eCombine \Coloneqq \eTAbs{\tVar_1}{\kType}{\eTAbs{\tVar_2}{\kWitness{\tMonoid{\tVar_1}}}{\eLet{\eMonoid{\wildcard}{\eVar}}{\eReflect{\tVar_2}}{\eVar}}} \]
    We create an instance of the class by invoking the data constructor and reifying the result.
    \[ \eReify{\eMonoid{0}{\eAddInt}}{\tVar}{\eApp{\eApp{\eCombine}{3}}{4}} \]

  \subsection{Effects and handlers}
