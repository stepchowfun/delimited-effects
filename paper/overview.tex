\section{Overview}

  \subsection{Reflection and reification}

    Here's a simple example of reification and reflection.

    \noindent
    \begin{minipage}{\linewidth}
      \begin{lstlisting}[gobble=8, mathescape=true]
        reify "Hello" as a in
          x $\leftarrow$ reflect a in
            x + ", World!"
      \end{lstlisting}
    \end{minipage}

  \subsection{Type classes with local instances}

    We can represent a type class as an ordinary data type.

    \noindent
    \begin{minipage}{\linewidth}
      \begin{lstlisting}[gobble=8, mathescape=true]
        data Monoid a = Monoid a (a $\rightarrow$ a $\rightarrow$ a)
      \end{lstlisting}
    \end{minipage}

    \noindent For convenience, we define \emph{methods} for the class. These are functions that reflect a reified instance back into a term and extract the appropriate fields.

    \noindent
    \begin{minipage}{\linewidth}
      \begin{lstlisting}[gobble=8, mathescape=true]
        identity : <Monoid a> $\Rightarrow$ a
        identity = $\lambda$(a : *) (b : <Monoid a>) $\rightarrow$
          let Monoid x _ = reflect b in x

        combine : <Monoid a> $\Rightarrow$ a $\rightarrow$ a $\rightarrow$ a
        combine = $\lambda$(a : *) (b : <Monoid a>) $\rightarrow$
          let Monoid _ x = reflect b in x
      \end{lstlisting}
    \end{minipage}

    \noindent We create an instance of the class by invoking the data constructor and reifying the result.

    \noindent
    \begin{minipage}{\linewidth}
      \begin{lstlisting}[gobble=8, mathescape=true]
        reify Monoid 0 addInt as a in
          combine 3 4
      \end{lstlisting}
    \end{minipage}

  \subsection{Effects and handlers}
