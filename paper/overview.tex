\section{Overview}

  \subsection{Reflection and reification}

    Here's a simple example of reification and reflection.
    \[ \eReify{\tVar}{\numberName{4}}{\parens{\numberName{3} + \eReflect{\tVar}}} \]

  \subsection{Type classes with local instances}

    We can represent a type class as an ordinary data type.
    \[ \eData{\tMonoid{\tVar}}{\eMonoid{\tVar}{\parens{\tArrow{\tVar}{\tArrow{\tVar}{\tVar}}}}} \]
    For convenience, we define \emph{methods} for the class. These are functions that reflect a reified instance and project the appropriate components.
    \[ \eIdentity = \eTAbs{\tVar_1}{\kType}{\eTAbs{\tVar_2}{\kWitness{\tMonoid{\tVar_1}}}{\eLet{\eMonoid{\eVar}{\wildcard}}{\eReflect{\tVar_2}}{\eVar}}} \]
    \[ \eCombine = \eTAbs{\tVar_1}{\kType}{\eTAbs{\tVar_2}{\kWitness{\tMonoid{\tVar_1}}}{\eLet{\eMonoid{\wildcard}{\eVar}}{\eReflect{\tVar_2}}{\eVar}}} \]
    Here's a polymorphic function which ``squares'' its argument using the monoid operation:
    \[ \eSquare = \eTAbs{\tVar_1}{\kType}{\eTAbs{\tVar_2}{\kWitness{\tMonoid{\tVar_1}}}{\eAbs{\eVar}{\tVar_1}{\eApp{\eApp{\eTApp{\eTApp{\eCombine}{\tVar_1}}{\tVar_2}}{\eVar}}{\eVar}}}} \]
    Here, we create and use a local instance of the class which reifies the monoidal structure of integers under multiplication:
    \[ \eReify{\tVar}{\eMonoid{\numberName{1}}{\eMultiply}}{\eApp{\eSquare}{\numberName{3}}} \]

  \subsection{Effects and handlers}
