%%%%%%%%%%%%%%%%%%%%%
% Document settings %
%%%%%%%%%%%%%%%%%%%%%

\documentclass[12pt]{article}
\usepackage[margin=1in]{geometry} % Just for setting the margin

\title{Delimited effects}
\date{}

% Add a "Draft" watermark to the background
\usepackage{draftwatermark}
\SetWatermarkText{\textsc{Draft}}
\SetWatermarkScale{3}

%%%%%%%%%%%%
% Packages %
%%%%%%%%%%%%

\usepackage[colorlinks]{hyperref}
\usepackage{amsmath}
\usepackage{amssymb}
\usepackage{bussproofs}
\usepackage{mathtools}
\usepackage{mdframed}
\usepackage{stackengine} % For stacking axioms

%%%%%%%%%%
% Macros %
%%%%%%%%%%

% Misc
\newcommand\parens[1]{\left( #1 \right)} % chktex 37
\newcommand\sub[3]{#1 \left[ #2 \mapsto #3 \right]} % chktex 1

% Terms
\newcommand\eterm{t}
\newcommand\eunit{()}
\newcommand\evar{x}
\newcommand\eabs[2]{\lambda #1 \; . \; #2} % chktex 1 chktex 26
\newcommand\eapp[2]{#1 \; #2}
\newcommand\eeffect[4]{\textbf{effect} \; #1 \; \textbf{with} \; \tanno{#2}{#3} \; \textbf{in} \; #4}
\newcommand\eprovide[4]{\textbf{provide} \; #1 \; \textbf{with} \; #2 = #3 \; \textbf{in} \; #4}
\newcommand\etabs[2]{\lambda #1 \; . \; #2} % chktex 1 chktex 26
\newcommand\etapp[2]{#1 \; \left[#2\right]}

% Types
\newcommand\ttype{\tau}
\newcommand\tunit{1}
\newcommand\tvar{\alpha}
\newcommand\tarrow[2]{#1 \rightarrow #2} % chktex 1
\newcommand\tanno[2]{#1 : #2} % chktex 26
\newcommand\tjudgment[4]{#1; #2 \vdash \tanno{#3}{#4}} % chktex 1
\newcommand\tx{\sigma}
\newcommand\twithx[2]{#1 \; ! \; #2} % chktex 26
\newcommand\tsub[3]{#1\left[#2/#3\right]}
\newcommand\tforall[2]{\forall #1 \; . \; #2} % chktex 1 chktk 1 chktex 26

% Effects
\newcommand\xeffect{e}
\newcommand\xeffects{E}
\newcommand\xempty{\varnothing_{\xeffect}}
\newcommand\xextend[2]{#1, #2}
\newcommand\xunion[2]{#1, #2}
\newcommand\xnotint[2]{#1 \notin #2} % chktex 1
\newcommand\xopwellformed[2]{#1 \vdash #2} % chktex 1

% Type contexts
\newcommand\ccontext{\Gamma}
\newcommand\cempty{\varnothing_{\ccontext}}
\newcommand\cextend[2]{#1, #2}
\newcommand\cunion[2]{#1, #2}

% Effect contexts
\newcommand\dcontext{\Delta}
\newcommand\dempty{\varnothing_{\dcontext}}
\newcommand\dextend[2]{#1, #2}
\newcommand\dunion[2]{#1, #2}
\newcommand\deffect[3]{#1 \mapsto \tanno{#2}{#3}} % chktex 1

%%%%%%%%%%%%%%%%%%%%%
% The specification %
%%%%%%%%%%%%%%%%%%%%%

\begin{document}

  \maketitle

  \begin{abstract}
    We introduce delimited effects, a modest type and effect system capable of modeling a diverse collection of programming language features including algebraic effects, type classes, and dynamic scoping.
  \end{abstract}

  \section{Introduction}

  \begin{figure}
    \begin{mdframed}
      \begin{center}
        \begin{tabular}{l l l}
          $\eterm \Coloneqq $ & & term \\
          & $\eunit$ & unit \\
          & $\evar$ & variable \\
          & $\eabs{\tanno{\evar}{\tx}}{\eterm}$ & abstraction \\
          & $\eapp{\eterm}{\eterm}$ & effect application \\
          & $\etabs{\tvar}{\eterm}$ & type abstraction \\
          & $\etapp{\eterm}{\tx}$ & type application \\
          & $\eeffect{\xeffect}{\evar}{\tx}{\eterm}$ & effect declaration \\
          & $\eprovide{\xeffect}{\evar}{\eterm}{\eterm}$ & effect definition \\
          $\ttype \Coloneqq$ & & type \\
          & $\tunit$ & unit type \\
          & $\tvar$ & type variable \\
          & $\tarrow{\tx}{\tx}$ & arrow type \\
          & $\tforall{\tvar}{\tx}$ & universal type \\
          $\tx \Coloneqq$ & & type with effects \\
          & $\twithx{\ttype}{\xeffects}$ & effect annotation \\
          $\xeffects \Coloneqq$ & & effect set \\
          & $\xempty$ & empty effect \\
          & $\xextend{\xeffects}{\xeffect}$ & effect extension \\
          $\ccontext \Coloneqq$ & & type context \\
          & $\cempty$ & empty type context \\
          & $\cextend{\ccontext}{\tanno{\evar}{\tx}}$ & variable binding \\
          & $\cextend{\ccontext}{\tvar}$ & type variable binding \\
          $\dcontext \Coloneqq$ & & effect context \\
          & $\dempty$ & empty effect context \\
          & $\dextend{\dcontext}{\deffect{\xeffect}{\evar}{\tx}}$ & effect binding \\
        \end{tabular}
      \end{center}

      \bigskip

      Let $\xunion{\xeffects_1}{\xeffects_2}$ denote the concatenation of $\xeffects_1$ and $\xeffects_2$. Formally, let $\xunion{\xeffects}{\xempty} = \xeffects$ and $\xunion{\xeffects_1}{\parens{\xextend{\xeffects_2}{\xeffect}}} = \xextend{\parens{\xunion{\xeffects_1}{\xeffects_2}}}{\xeffect}$. We use analogous notation for $\ccontext$ and $\dcontext$. Let $\xnotint{\xeffect}{\tx}$ denote the judgment that $\xeffect$ does not appear anywhere in the type or effects of $\tx$.

      \caption{Syntax}\label{fig:syntax}
    \end{mdframed}
  \end{figure}

  \begin{figure}
    \begin{mdframed}
      \begin{center}
        \framebox{$\xopwellformed{\xeffect}{\tx}$}
      \end{center}

      \medskip

      \begin{prooftree}
          \AxiomC{}
        \RightLabel{(\textsc{O-Unit})}
        \UnaryInfC{$\xopwellformed{\xeffect}{\twithx{\tunit}{\xextend{\xeffects}{\xeffect}}}$}
      \end{prooftree}

      \begin{prooftree}
          \AxiomC{$\xopwellformed{\xeffect}{\tx_2}$}
        \RightLabel{(\textsc{O-Arrow})}
        \UnaryInfC{$\xopwellformed{\xeffect}{\twithx{\parens{\tarrow{\tx_1}{\tx_2}}}{\xeffects}}$}
      \end{prooftree}

      \caption{Operation type well-formedness}\label{fig:operation_type}
    \end{mdframed}
  \end{figure}

  \begin{figure}
    \begin{mdframed}
      \begin{center}
        \framebox{$\tjudgment{\ccontext}{\dcontext}{\eterm}{\tx}$}
      \end{center}

      \medskip

      \begin{prooftree}
          \AxiomC{}
        \RightLabel{(\textsc{T-Unit})}
        \UnaryInfC{$\tjudgment{\ccontext}{\dcontext}{\eunit}{\twithx{\tunit}{\xempty}}$}
      \end{prooftree}

      \begin{prooftree}
          \AxiomC{}
        \RightLabel{(\textsc{T-Variable})}
        \UnaryInfC{$\tjudgment{\cextend{\ccontext}{\tanno{\evar}{\tx}}}{\dcontext}{\evar}{\tx}$}
      \end{prooftree}

      \begin{prooftree}
          \AxiomC{$\tjudgment{\cextend{\ccontext}{\tanno{\evar}{\tx_1}}}{\dcontext}{\eterm}{\tx_2}$}
        \RightLabel{(\textsc{T-Abstraction})}
        \UnaryInfC{$\tjudgment{\ccontext}{\dcontext}{\eabs{\tanno{\evar}{\tx_1}}{\eterm}}{\tarrow{\tx_1}{\tx_2}}$}
      \end{prooftree}

      \begin{prooftree}
          \AxiomC{$\tjudgment{\ccontext}{\dcontext}{\eterm_1}{\twithx{\ttype_1}{\xeffects_1}}$}
          \AxiomC{$\tjudgment{\ccontext}{\dcontext}{\eterm_2}{\twithx{\parens{\tarrow{\twithx{\ttype_1}{\xeffects_4}}{\twithx{\ttype_2}{\xeffects_2}}}}{\xeffects_3}}$}
        \RightLabel{(\textsc{T-Application})}
        \BinaryInfC{$\tjudgment{\ccontext}{\dcontext}{\eapp{\eterm_2}{\eterm_1}}{\twithx{\ttype_2}{\xunion{\xeffects_1}{\xunion{\xeffects_2}{\xeffects_3}}}}$}
      \end{prooftree}

      \begin{prooftree}
          \AxiomC{$\tjudgment{\ccontext}{\dextend{\dcontext}{\deffect{\xeffect}{\evar}{\tx_1}}}{\eterm_2}{\twithx{\ttype}{\xeffects_1}}$}
          \AxiomC{$\tjudgment{\ccontext}{\dextend{\dcontext}{\deffect{\xeffect}{\evar}{\tx_1}}}{\eterm_1}{\tx_2}$}
          \AxiomC{$\tsub{\tx_1}{\xeffect}{\xeffects_2} = \tx_2$}
          \AxiomC{$\xnotint{e}{\ttype}$}
        \RightLabel{(\textsc{T-Provide})}
        \QuaternaryInfC{$\tjudgment{\ccontext}{\dextend{\dcontext}{\deffect{\xeffect}{\evar}{\tx_1}}}{\eprovide{\xeffect}{\evar}{\eterm_1}{\eterm_2}}{\twithx{\ttype}{\parens{\xextend{\xeffects_1}{\xeffects_2}} - e}}$}
      \end{prooftree}

      \begin{prooftree}
          \AxiomC{$\tjudgment{\cextend{\ccontext}{\tanno{\evar}{\tx_1}}}{\dextend{\dcontext}{\deffect{\xeffect}{\evar}{\tx_1}}}{\eterm}{\tx_2}$}
          \AxiomC{$\xopwellformed{\xeffect}{\tx_1}$}
          \AxiomC{$\xeffect \notin \dcontext$}
          \AxiomC{$\xeffect \notin \tx_2$}
        \RightLabel{(\textsc{T-Effect})}
        \QuaternaryInfC{$\tjudgment{\ccontext}{\dcontext}{\eeffect{\xeffect}{\evar}{\tx_1}{\eterm}}{\tx_2}$}
      \end{prooftree}

      \caption{Typing rules}\label{fig:typing_rules}
    \end{mdframed}
  \end{figure}

  \begin{figure}
    \begin{mdframed}
      \begin{center}
        \framebox{$\tjudgment{\ccontext}{\dcontext}{\eterm}{\tx}$}
      \end{center}

      \medskip

      \begin{prooftree}
          \AxiomC{$\tjudgment{\ccontext}{\dcontext}{\eterm}{\tx_2}$}
        \RightLabel{(\textsc{$\ccontext$-Weaken})}
        \UnaryInfC{$\tjudgment{\cextend{\ccontext}{\tanno{x}{\tx_1}}}{\dcontext}{\eterm}{\tx_2}$}
      \end{prooftree}

      \begin{prooftree}
          \AxiomC{$\tjudgment{\cunion{\cextend{\cextend{\ccontext_1}{\tanno{x_1}{\tx_1}}}{\tanno{x_2}{\tx_2}}}{\ccontext_2}}{\dcontext}{\eterm}{\tx_3}$}
        \RightLabel{(\textsc{$\ccontext$-Permutation})}
        \UnaryInfC{$\tjudgment{\cunion{\cextend{\cextend{\ccontext_1}{\tanno{x_2}{\tx_2}}}{\tanno{x_1}{\tx_1}}}{\ccontext_2}}{\dcontext}{\eterm}{\tx_3}$}
      \end{prooftree}

      \begin{prooftree}
          \AxiomC{$\tjudgment{\ccontext}{\dcontext}{\eterm}{\twithx{\ttype}{\xeffects}}$}
        \RightLabel{(\textsc{$\xeffects$-Weaken1})}
        \UnaryInfC{$\tjudgment{\ccontext}{\dcontext}{\eterm}{\twithx{\ttype}{\xextend{\xeffects}{\xeffect}}}$}
      \end{prooftree}

      \begin{prooftree}
          \AxiomC{$\tjudgment{\cextend{\ccontext}{\tanno{x}{\twithx{\ttype}{\xextend{\xeffects}{\xeffect}}}}}{\dcontext}{\eterm}{\tx}$}
        \RightLabel{(\textsc{$\xeffects$-Weaken2})}
        \UnaryInfC{$\tjudgment{\cextend{\ccontext}{\tanno{x}{\twithx{\ttype}{\xeffects}}}}{\dcontext}{\eterm}{\tx}$}
      \end{prooftree}

      \begin{prooftree}
          \AxiomC{$\tjudgment{\ccontext}{\dcontext}{\eterm}{\twithx{\ttype}{\xunion{\xextend{\xextend{\xeffects_1}{\xeffect_1}}{\xeffect_2}}{\xeffects_2}}}$}
        \RightLabel{(\textsc{$\xeffects$-Permutation1})}
        \UnaryInfC{$\tjudgment{\ccontext}{\dcontext}{\eterm}{\twithx{\ttype}{\xunion{\xextend{\xextend{\xeffects_1}{\xeffect_2}}{\xeffect_1}}{\xeffects_2}}}$}
      \end{prooftree}

      \begin{prooftree}
          \AxiomC{$\tjudgment{\cextend{\ccontext}{\tanno{x}{\twithx{\ttype}{\xunion{\xextend{\xextend{\xeffects_1}{\xeffect_1}}{\xeffect_2}}{\xeffects_2}}}}}{\dcontext}{\eterm}{\tx}$}
        \RightLabel{(\textsc{$\xeffects$-Permutation2})}
        \UnaryInfC{$\tjudgment{\cextend{\ccontext}{\tanno{x}{\twithx{\ttype}{\xunion{\xextend{\xextend{\xeffects_1}{\xeffect_2}}{\xeffect_1}}{\xeffects_2}}}}}{\dcontext}{\eterm}{\tx}$}
      \end{prooftree}

      \caption{Structural rules}\label{fig:structural_rules}
    \end{mdframed}
  \end{figure}

\end{document}
